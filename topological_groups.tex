% Document setup
\documentclass[article, a4paper, 11pt, oneside]{memoir}
\usepackage[utf8]{inputenc}
\usepackage[T1]{fontenc}
\usepackage[UKenglish]{babel}

% Document info
\newcommand\doctitle{Topological Groups and Vector Spaces}
\newcommand\docauthor{Danny Nygård Hansen}

% Formatting and layout
\usepackage[autostyle]{csquotes}
\usepackage[final]{microtype}
\usepackage{xcolor}
\frenchspacing
\usepackage{latex-sty/articlepagestyle}
\usepackage{latex-sty/articlesectionstyle}
% \usepackage{latex-sty/amalgsymbol}

% Fonts
\usepackage[largesmallcaps]{kpfonts}
\DeclareSymbolFontAlphabet{\mathrm}{operators} % https://tex.stackexchange.com/questions/40874/kpfonts-siunitx-and-math-alphabets
\linespread{1.06}
\let\mathfrak\undefined
\usepackage{eufrak}
\usepackage{inconsolata}
% \usepackage{amssymb}

% Hyperlinks
\usepackage{hyperref}
\definecolor{linkcolor}{HTML}{4f4fa3}
\hypersetup{%
    pdftitle=\doctitle,
    pdfauthor=\docauthor,
    colorlinks,
    linkcolor=linkcolor,
    citecolor=linkcolor,
    urlcolor=linkcolor,
    bookmarksnumbered=true
}

% Equation numbering
\numberwithin{equation}{chapter}

% Footnotes
\footmarkstyle{\textsuperscript{#1}\hspace{0.25em}}

% Mathematics
\usepackage{latex-sty/basicmathcommands}
\usepackage{latex-sty/framedtheorems}
\usepackage{latex-sty/topologycommands}
\usepackage{tikz-cd}
\tikzcdset{arrow style=math font} % https://tex.stackexchange.com/questions/300352/equalities-look-broken-with-tikz-cd-and-math-font
\usetikzlibrary{babel}

% Lists
\usepackage{enumitem}
\setenumerate[0]{label=\normalfont(\alph*)}
\setlist{
    listparindent=\parindent,
    parsep=0pt,
}

% Bibliography
\usepackage[backend=biber, style=authoryear, maxcitenames=2, useprefix]{biblatex}
\addbibresource{references.bib}

% Title
\title{\doctitle}
\author{\docauthor}

\newcommand{\calT}{\mathcal{T}}
\newcommand{\calB}{\mathcal{B}}
\newcommand{\calP}{\mathcal{P}}
\newcommand{\calM}{\mathcal{M}}
\newcommand{\calU}{\mathcal{U}}
\newcommand{\calL}{\mathcal{L}}
\newcommand{\calF}{\mathcal{F}}
\newcommand{\calH}{\mathcal{H}}
\newcommand{\calE}{\mathcal{E}}


% Section style -- add to section style .sty?
\setsubsecheadstyle{\normalfont\itshape}


\begin{document}

\maketitle

\chapter{Introduction}

The purpose of these notes is twofold: First of all to clarify some of the basic properties of topological groups. We in particular hope to shed light on the reasonability of the common assumption that topological groups be Hausdorff.

Secondly, we provide an overview of basic results about topological vector spaces. These are in particular topological groups, so we rely on the theory developed in the first part of these notes.

Most of the results below can be found in various forms in a variety of books or articles (see the references), but I have not come across a resource that develops this theory in precisely this way. One example is the $T_0$-identification of a topological group, as we will see below.


\chapter{Topological groups}

\section{Definitions and basic properties}

\begin{definition}[Topological groups]
    \label{def:topological_group}
    A \emph{topological group} is a triple $(G, \mu, \calT)$ such that $(G, \mu)$ is a group, $(G, \calT)$ is a topological space, and both the multiplication $\mu \colon G \prod G \to G$ and the inversion map $\iota \colon G \to G$ given by $g \mapsto g\inv$ are continuous.
\end{definition}
%
In the sequel we will simply write $G$, and it will be clear from context whether $G$ is to be thought of as a set, group, topological space or topological group. The identity on $G$ will be denoted $e_G$, or simply $e$. It is custom to assume that a topological group be $T_1$ (or Hausdorff, which is actually implied by it being $T_1$, as will follow from \cref{thm:topological_group_regular}), but we omit this assumption since our focus is precisely on the basic topological properties of topological groups.

\newcommand{\leftmult}{\lambda}
\newcommand{\rightmult}{\rho}
\newcommand{\conjmult}{\gamma}

\newcommand{\calN}{\mathcal{N}}
\DeclarePairedDelimiter{\nhoodfilteraux}{(}{)}
% \newcommand{\nhoodfilter}[1]{\calN\nhoodfilteraux{#1}}
\newcommand{\nhoodfilter}[1]{\calN_{#1}}

Let $G$ be a topological group and let $g \in G$. We define maps $\leftmult_g, \rightmult_g \colon G \to G$ by $\leftmult_g(h) = gh$ and $\rightmult_g(h) = hg$; that is, $\leftmult_g$ and $\rightmult_g$ are given by left- and right-multiplication by $g$, respectively. Both of these are homeomorphisms, with $\leftmult_g\inv = \leftmult_{g\inv}$ and $\rightmult_g\inv = \rightmult_{g\inv}$. Hence if $a, b \in G$, then there is a homeomorphism on $G$ that takes $a$ to $b$, namely $\leftmult_{ba\inv}$. Thus a topological group is \emph{homogeneous} as a topological space. In particular,\footnote{If $X$ is a topological space and $A \subseteq X$, then we say that a set $N \subseteq X$ is a \emph{neighbourhood} of $A$ if there is an open set $U$ in $X$ such that $A \subseteq U \subseteq N$. The family of neighbourhoods of a set $A$ is called the \emph{neighbourhood filter of $A$} and is denoted $\nhoodfilter{A}$. If $A = \{x\}$ is a singleton we also write $\nhoodfilter{x}$ and call $N$ a neighbourhood of $x$.} $\nhoodfilter{g} = \leftmult_g(\nhoodfilter{e}) = g\nhoodfilter{e}$ for all $g \in H$, so the topology on $G$ is determined by any neighbourhood basis at $e$. A neighbourhood basis at $e$ is called a \emph{local basis}.

If $G$ is a topological group and $A \subseteq G$, then we write
%
\begin{equation*}
    A\inv = \iota(A) = \set{a\inv}{a \in A}.
\end{equation*}
%
A subset $A$ is called \emph{symmetric} if $A = A\inv$. Notice that since $\iota$ is a homeomorphism, $A$ is open (closed) if and only if $A\inv$ is open (closed).

We begin by collecting some basic properties of topological groups:

\begin{proposition}[Properties of topological groups]
    Let $G$ be a topological group, and let $A, B \subseteq G$.
    %
    \begin{enumprop}
        \item \label{enum:closure_of_product} $\closure{A} \closure{B} \subseteq \closure{AB}$.

        \item \label{enum:product_open_set} If $U \subseteq G$ is open, then $AU$ and $UA$ are open.

        \item \label{enum:product-compact-closed} If $K \subseteq G$ is compact and $F \subseteq G$ is closed, then $KF$ and $FK$ are closed.

        \item \label{enum:closure-intersection-of-open-sets} $\closure{A} = \bigintersect_{U \in \nhoodfilter{0}} AU$.
        
        \item \label{enum:elements-of-closure-commute} Assume that $G$ is Hausdorff. If $ab = ba$ for all $a \in A$, $b \in B$, then $ab = ba$ for all $a \in \closure{A}$, $b \in \closure{B}$.
    \end{enumprop}
\end{proposition}
%
As we will see in \cref{thm:topological_group_regular}, the Hausdorff assumption in \subcref{enum:elements-of-closure-commute} can be weakened to just $T_0$. However, this assumption cannot be dropped: Any non-trivial, nonabelian group is a topological group in the trivial topology, but then the closure of the trivial subgroup is the whole group and hence not abelian.

\begin{proof}
    We give two proofs of \subcref{enum:closure_of_product}. Since the multiplication $\mu \colon G \prod G \to G$ is continuous we get
    %
    \begin{equation*}
        \mu(\closure{A}, \closure{B})
            = \mu(\closure{A} \prod \closure{B})
            = \mu(\closure{A \prod B})
            \subseteq \closure{\mu(A \times B)}
            = \closure{\mu(A, B)}
            = \closure{AB}.
    \end{equation*}
    %
    Alternatively, consider $a \in \closure{A}$ and $b \in \closure{B}$. If $U$ is a neighbourhood of $ab$, then by continuity of multiplication there are neighbourhoods $V_1$ and $V_2$ of $a$ and $b$ respectively such that $V_1 V_2 \subseteq U$. Picking $x \in A \intersect V_1$ and $y \in B \intersect V_2$ we find that $xy \in AB \intersect U$, so $ab \in \closure{AB}$.

    To prove \subcref{enum:product_open_set}, notice that e.g.
    %
    \begin{equation*}
        AU = \bigunion_{g \in A} gU,
    \end{equation*}
    %
    so $AU$ is a union of open sets.

    For \subcref{enum:product-compact-closed}, consider the map $\phi \colon K \times G \to G$ given by $\phi(x,y) = x\inv y$. If $g \not\in KF$, then $K \times \{g\} \subseteq \phi\preim(F^c)$. The tube lemma then furnishes an open neighbourhood $U$ of $g$ such that $K \times U \subseteq \phi\preim(F^c)$, which implies that $U \intersect KF = \emptyset$. That $FK$ is closed is proved similarly.

    To prove \subcref{enum:closure-intersection-of-open-sets}, notice that $g \in \closure{A}$ if and only if $A \intersect V \neq \emptyset$ for all $V \in \nhoodfilter{g}$, if and only if $A \intersect gU \neq \emptyset$ for all $U \in \nhoodfilter{0}$. Since $\nhoodfilter{0}$ is symmetric, this is the case just when $g \in AU$ for all $U \in \nhoodfilter{0}$.

    Let $\conjmult_a \colon G \to G$ be conjugation by $a$, i.e. $\conjmult_a(g) = aga\inv$. Then \subcref{enum:elements-of-closure-commute} says that if $\conjmult_a$ is the identity map on $B$, then it is the identity map on $\closure{B}$. But since $G$ is Hausdorff, this follows.
\end{proof}


\section{Separability in topological groups}

% We next remark that the assumption that $G$ be $T_1$ can be weakened:

% \begin{proposition}[$T_0$ implies $T_1$]
%     \label{thm:T0_implies_T1}
%     Let $G$ be a topological group. If $G$ is $T_0$, then it is in fact $T_1$.
% \end{proposition}

% \begin{proof}
%     Assume that $G$ is $T_0$. By homogeneity it suffices to show that the singleton $\{e\}$ containing the identity $e \in G$ is closed. We show that $G \setminus \{e\}$ is a neighbourhood of all its points. Let $g \neq e$. Since $G$ is $T_0$, either $G \setminus \{e\}$ is a neighbourhood of $g$ or $G \setminus \{g\}$ is a neighbourhood of $e$. In the first case we are done, so assume the latter. The homeomorphism $\leftmult_{g\inv}$ maps $G \setminus \{g\}$ to $G \setminus \{e\}$, so the latter set is a neighbourhood of $g\inv$. But the inversion map $\iota$ is a homeomorphism that maps $G \setminus \{e\}$ to itself, so this is also a neighbourhood of $g$. This proves the claim.
% \end{proof}

Recall that a topological space $X$ is \emph{regular} if it is possible to separate any point from any closed set by disjoint open sets.\footnote{In our terminology, a regular Hausdorff space would be called a \emph{$T_3$-space}.} Our next order of business is to show that any topological group is regular.\footnote{In fact, topological groups are \emph{completely regular}. The proofs I know of this fact uses the theory of uniform spaces, so we do not cover it in these notes.} Before proving this we need a lemma:

\begin{lemma}
    \label{thm:symmetric_nhood_of_e}
    Let $G$ be a topological group. If $U$ is an open neighbourhood of the identity $e$, then there is a symmetric open neighbourhood $V$ of $e$ such that $VV \subseteq U$. In particular, $G$ has a local basis of symmetric open sets.
\end{lemma}

\begin{proof}
    Since multiplication is continuous and $ee = e$, there are open neighbourhoods $V_1$ and $V_2$ of $e$ such that $V_1 V_2 \subseteq U$. If we let
    %
    \begin{equation*}
        V = V_1 \intersect V_2 \intersect V_1\inv \intersect V_2\inv,
    \end{equation*}
    %
    then $V$ has the desired properties. The final claim follows from the fact that $V \subseteq VV$ since $e \in V$.
\end{proof}



\begin{proposition}[Regularity]
    If $G$ is a topological group, $K \subseteq G$ is compact and $F \subseteq G$ is closed, then there exists a symmetric open neighbourhood of $e$ such that $VK$ and $VF$ are disjoint. In particular, $G$ is regular.
\end{proposition}

\begin{proof}
    Since $K$ and $F$ are disjoint, we have $e \not\in KF\inv$. But $KF\inv$ is closed by \cref{enum:product-compact-closed}, so \cref{thm:symmetric_nhood_of_e} yields a symmetric open neighbourhood $V$ of $e$ such that $VV \intersect KF\inv = \emptyset$. This implies that $VK$ and $VF$ are disjoint as desired.
\end{proof}



\begin{lemma}
    \label{lem:complete-regularity-lemma}
    Let $G$ be a topological group. If $F \subseteq G$ is closed and $g \in G \setminus F$, then there exists a continuous function $f \colon G \to [0,1]$ such that $f(g) = 1$ and $f(F) = \{0\}$. Furthermore, given $\epsilon > 0$ there exists a neighbourhood $U$ of $e$ such that $\abs{f(gx) - f(x)} \leq \epsilon$ for all $g \in U$ and $x \in G$.

    If $G$ is first countable, then the sets $\{f > 1 - \tfrac{1}{n}\}$ for $n \in \naturals$ form a neighbourhood basis at $g$.
\end{lemma} % https://terrytao.wordpress.com/2011/05/17/the-birkhoff-kakutani-theorem/

\begin{proof}
    By homogeneity we may assume that $g = e$. If $G$ is first countable, then we choose a decreasing sequence $(V_n)_{n \in \naturals}$ of open neighbourhoods of $e$ that constitute a local basis.

    We recursively construct a decreasing sequence $(U_{1/2^n})_{n \in \naturals_0}$ of open neighbourhoods of $e$. First choose $U_1$ to be disjoint from $F$ and symmetric in accordance with \cref{thm:symmetric_nhood_of_e}. The same lemma is applied recursively such that each $U_{1/2^n}$ is symmetric and $U_{1/2^{n+1}} U_{1/2^{n+1}} \subseteq U_{1/2^n}$ for all $n \in \naturals_0$. If $G$ is first countable, we also choose the sets such that $U_{1/2^n} \subseteq V_n$, making the sequence into a local basis.

    Next, for $a/2^n \in (0,1)$ let $a/2^n = 2^{-n_1} + \cdots + 2^{-n_k}$ be the binary expansion of $a/2^n$, with $1 \leq n_1 < \ldots < n_k$. We then let
    %
    \begin{equation*}
        U_{a/2^n}
            = U_{1/2^{n_k}} \cdots U_{1/2^{n_1}}.
    \end{equation*}
    %
    It follows [TODO] that
    %
    \begin{equation*}
        U_{1/2^n} U_{a/2^n}
            \subseteq U_{(a+1)/2^n}
    \end{equation*}
    %
    for all $n \in \naturals_0$ and $a \in [1,2^n)$ (note that the case $n = 0$ is trivial, since then there is no such $a$). Finally define $f \colon G \to [0,1]$ by
    %
    \begin{equation*}
        f(x)
            = \sup \set{1 - \tfrac{a}{2^n}}{n \in \naturals_0, a \in [1,2^n), x \in U_{a/2^n}},
    \end{equation*}
    %
    with the convention that $\sup \emptyset = 0$. Then $f$ is uniformly continuous in the sense described above, and the neighbourhood basis property follows. [TODO details]
\end{proof}


\begin{proposition}
    Every topological group is completely regular.
\end{proposition}

\begin{proof}
    This follows immediately from \cref{lem:complete-regularity-lemma}.
\end{proof}



\section{Continuous group homomorphisms}

% Let $G$ and $H$ be topological groups. A map $\phi \colon G \to H$ is called a \emph{topological (group) homomorphism} if it is a continuous group homomorphism. If $\phi$ is bijective and its inverse is continuous, then it is called a \emph{topological isomorphism}. That is, if $\phi$ is both a group homomorphism and a homeomorphism.

We next explore the relationship between the topological and algebraic structure for maps. Let $A \subseteq G$. A map $f \colon A \to H$ is said to be \emph{uniformly continuous} if there for each neighbourhood $V$ of $e_H$ exists a neighbourhood $U$ of $e_G$ such that $x\inv y \in U$ implies $f(x)\inv f(y) \in V$, for all $x,y \in A$. Uniform continuity clearly implies continuity, since the above says that $f(yU) \subseteq f(y)V$, and every neighbourhood of $y$ and $f(y)$ are on the form $yU$ respectively $f(y)V$ by homogeneity.

If $f$ is injective and both $f \colon A \to f(A)$ and $f\inv \colon f(A) \to A$ are uniformly continuous, then $f$ is called a \emph{unimorphism}. Notice that $f$ need not be surjective for it to be a unimorphism. Notice also that any restriction of a unimorphism is also a unimorphism, since the restriction of a uniformly continuous map is uniformly continuous.

As expected from homogeneity, all topological homomorphisms are automatically uniformly continuous:

\begin{proposition}
    \label{thm:group-homomorphism-continuity}
    Let $\phi \colon G \to H$ be a group homomorphism between topological groups that is continuous at some point $g \in G$. Then $\phi$ is uniformly continuous.
\end{proposition}

\begin{proof}
    We may assume that $\phi$ is continuous at $e_G$, since
    %
    \begin{equation*}
        \phi
            = \phi \circ \leftmult_{g\inv} \circ \leftmult_{g}
            = \leftmult_{\phi(g)\inv} \circ \phi \circ \leftmult_{g},
    \end{equation*}
    %
    and the map on the right-hand side is continuous at $e_G$ iff $\phi$ is continuous at $g$.\footnote{This already shows that $\phi$ is continuous.} Thus let $V$ be a neighbourhood of $e_H$. By continuity of $\phi$ at $e_G$, there exists a neighbourhood $U$ of $e_G$ such that $f(U) \subseteq V$. For $x,y \in G$ such that $x\inv y \in U$ we thus have
    %
    \begin{equation*}
        f(x)\inv f(y)
            = f(x\inv y)
            \in V,
    \end{equation*}
    %
    as required for uniform continuity.
\end{proof}


\section{Subgroups}

If $G$ is a topological group and $H$ is a subgroup of $G$, then $H$ is equipped with the subspace topology from $G$. Clearly the inherited group operations are still continuous, so $H$ is a topological group.

\begin{proposition}[Properties of subgroups]
    Let $G$ be a topological group.
    %
    \begin{enumprop}
        \item \label{enum:closure_of_subgroup} If $H$ is a subgroup of $G$, then so is $\closure{H}$. If $H$ is normal in $G$, then so is $\closure{H}$. If $G$ is $T_0$ and $H$ is abelian, then so is $\closure{H}$.
        
        \item \label{enum:open_subgroup_is_closed} Every open subgroup of $G$ is closed.
    \end{enumprop}
\end{proposition}

\begin{proof}
    Let $H$ be a subgroup of $G$. By \cref{enum:closure_of_product} we get
    %
    \begin{equation*}
        \closure{H} \closure{H} \subseteq \closure{HH} = \closure{H},
    \end{equation*}
    %
    so $\closure{H}$ is closed under multiplication. It is also closed under taking inverses, since the inverse map $\iota$ is a homeomorphism. Hence it is a subgroup. Clearly $\closure{H}$ is the smallest closed subgroup that contains $H$. So if $H$ is normal and $\closure{H}$ is not, then intersecting $\closure{H}$ with one of its conjugates yields a strictly smaller closed subgroup containing $H$, a contradiction.

    If $G$ is $T_0$ and $H$ is abelian, then it follows from \cref{enum:elements-of-closure-commute} with $A = B = H$ that $\closure{H}$ is abelian. This proves \subcref{enum:closure_of_subgroup}.

    Finally let $H$ be an open subgroup. Since $G$ is the disjoint union of the left cosets of $H$, we have
    %
    \begin{equation*}
        G \setminus H
            = \bigunion_{g \in G \setminus H} gH.
    \end{equation*}
    %
    Since the cosets $gH$ are open, $G \setminus H$ is also open. This proves \subcref{enum:open_subgroup_is_closed}.
\end{proof}


\section{Convergence and completeness}

% Overwrite \ball from topologycommands.sty
\renewcommand{\ball}[3][]{B_{#1}(#2,#3)}
\newcommand{\cball}[3][]{\overline{B}_{#1}(#2,#3)}
\newcommand{\pball}[3][]{B'_{#1}(#2,#3)}

If $P$ and $Q$ are preordered sets, recall that the product order on $P \prod Q$ is given by $(p,q) \leq (p',q')$ iff $p \leq p'$ and $q \leq q'$. If $P$ and $Q$ are directed, then this induces a direction on $P \prod Q$.

Let $G$ be a topological group, and let $(g_i)_{i \in I}$ be a net in $G$. Recall that $g_i$ converges to $g \in G$ if for every neighbourhood $U$ of $e$, $g_i$ eventually lies in $U$. Furthermore, $(g_i)$ is called a \emph{Cauchy net} if the net $(g_i\inv g_j)_{(i,j) \in I \prod I}$ converges to $e$. That is, if for every neighbourhood $U$ of $0$ there is an $i_0 \in I$ such that $i,j \geq i_0$ implies that $g_i\inv g_j \in U$. It is easy to see that a convergent net is Cauchy since the group operations are continuous. Conversely, we say that a subset $A \subseteq G$ is \emph{complete} if every Cauchy net in $A$ converges (in $G$) to a point in $A$.

Finally, we say that a metric $\rho$ on $G$ is \emph{left-invariant} if $\rho(ag,ah) = \rho(g,h)$ for all $a,g,h \in G$. Right-invariance is defined analogously.

\begin{remark}
    Assume that the topology on $(G,\calT)$ is generated by a pseudometric $\rho$. Then $\calT$-Cauchy nets do not necessarily correspond to $\rho$-Cauchy nets. For instance, consider the metric $\rho(x,y) = \abs{\arctan x - \arctan y}$ on $\reals$, i.e. the metric induced by the homeomorphism $\arctan \colon \reals \to (-\pi/2, \pi/2)$, where the latter interval is considered as a metric subspace of $\reals$ with the usual metric $d$. Then the net $(i)_{i \in \naturals}$ is Cauchy in $(\reals, \rho)$, but not Cauchy in $(\reals, d)$.

    However, if $\rho$ is either left- or right-invariant then the two notions coincide: This follows easily from the observation that in this case $g\inv h \in \ball[\rho]{e}{\epsilon}$ if and only if $\rho(g,h) < \epsilon$, for all $g,h \in G$ and $\epsilon > 0$.
\end{remark}


\begin{proposition}[Complete implies closed]
    \label{thm:complete-implies-closed}
    Let $G$ be a Hausdorff topological group. Then every complete subset of $G$ is closed in $G$.
\end{proposition}

\begin{proof}
    Let $A$ be a complete subset of $G$, and let $g \in \closure{A}$. Then there is a net $(g_i)_{i \in I}$ in $A$ that converges to $g$. This is then a Cauchy net in $A$, hence converges to some $g' \in A$. Since $G$ is Hausdorff limits are unique, so $g = g' \in A$.
\end{proof}


\begin{lemma}
    Let $G$ and $H$ be topological groups, let $A \subseteq G$, and let $f \colon A \to H$ be uniformly continuous. If $(g_i)_{i \in I}$ is a Cauchy net in $A$, then $(f(g_i))_{i \in I}$ is a Cauchy net.
\end{lemma}

\begin{proof}
    Let $V$ and $U$ be neighbourhoods as in the discussion above. Let $i_0 \in I$ be such that $i,j \geq i_0$ implies $g_i\inv g_j \in U$. Uniform continuity then implies that $f(g_i)\inv f(g_j) \in V$, and since $V$ was arbitrary the claim follows.
\end{proof}


\begin{theorem}
    \label{thm:completeness-preserved-by-unimorphism}
    Let $G$ and $H$ be topological groups, let $A \subseteq G$, and let $f \colon A \to H$ be a unimorphism. If $A$ is complete, then $B = f(A)$ is also complete.

    In particular, completeness is preserved by topological group isomorphisms.
\end{theorem}
%
We in fact only need $f$ to be continuous and to have a uniformly continuous right-inverse.

\begin{proof}
    Let $(h_i)_{i \in I}$ be a Cauchy net in $B$. Then since $f\inv \colon B \to A$ is uniformly continuous, $(f\inv(h_i))_{i \in I}$ is a Cauchy net in $A$, hence convergent to some $g \in A$ by completeness. Since $f$ is continuous, we have $h_i = f(f\inv(h_i)) \to f(g) \in B$, so $(h_i)$ converges to a point in $B$. Thus $B$ is complete.
\end{proof}


\section{Coset spaces and quotient groups}

If $H$ is a subgroup of a topological group $G$, we denote by $G/H$ the set of left cosets of $H$. Let $q \colon G \to G/H$ be the quotient map and equip $G/H$ with the quotient topology. We call $G/H$ a \emph{(left) coset space} of $G$ by $H$.

Notice that $q$ is in fact open: If $U \subseteq G$ is open, then $q\preim(q(U)) = UH$ is also open by \cref{enum:product_open_set}, so $q(U)$ is open since $G/H$ has the quotient topology coinduced by $q$.


\begin{proposition}[Properties of coset spaces]
    Let $G$ be a topological group and $H$ a subgroup.
    %
    \begin{enumprop}
        \item \label{enum:coset_space_regular} The coset space $G/H$ is regular.

        \item \label{enum:coset-space-homogeneous} The topology on $G/H$ is homogeneous.
        
        \item \label{enum:coset_space_T1} $G/H$ is $T_1$ (and hence $T_3$) if and only if $H$ is closed.

        \item \label{enum:coset-space-discrete} $G/H$ is discrete if and only if $H$ is open.
    \end{enumprop}
\end{proposition}
%
Notice that \subcref{enum:coset_space_T1} does not assume any separation properties of $G$.

% (The claim \subcref{enum:coset_space_regular} is the content of supplementary exercise 7(d) of \textcite[Chapter~2]{munkres}, except that in the exercise $H$ is assumed to be closed. I do not see where this assumption is used.)

\begin{proof}
\begin{proofsec}
    \item[Proof of \subcref{enum:coset_space_regular}]
    Let $q(x) \in G/H$, and let $B \subseteq G/H$ be a closed set. Then $A = q\preim(B)$ is closed, so by \cref{thm:topological_group_regular} there is a symmetric neighbourhood $V$ of $e$ in $G$ such that $Vx \intersect VA = \emptyset$. Since $A$ is a union of left cosets of $H$ we have $A = AH$, so $Vx \intersect VAHH = \emptyset$. Since $H$ is symmetric, it follows that
    %
    \begin{equation*}
        q(Vx) \intersect q(VA)
            = VxH \intersect VAH
            = \emptyset.
    \end{equation*}
    %
    Since $q$ is open the sets $q(Vx)$ and $q(VA)$ are neighbourhoods of $q(x)$ and $q(A) = B$ respectively.

    \item[Proof of \subcref{enum:coset-space-homogeneous}]
    For $g \in G$ define a map $\theta_g \colon G/H \to G/H$ by $\theta(q(x)) = q(gx)$. This is well-defined, since if $xH = yH$, then $gxH = gyH$. Furthermore, the diagram
    %
    \begin{equation*}
        \begin{tikzcd}
            G
                \ar[r, "\leftmult_g"]
                \ar[d, "q", swap]
            & G
                \ar[d, "q"] \\
            G/H
                \ar[r, "\theta_g", swap]
            & G/H
        \end{tikzcd}
    \end{equation*}
    %
    commutes, so $\theta_g \circ q$ is continuous. But by the characteristic property of the quotient topology on $G/H$, $\theta_g$ is also continuous. Since $\theta_g\inv = \theta_{g\inv}$ it is in fact a homeomorphism, and $\theta_{yx\inv}$ takes $q(x)$ to $q(y)$, which shows homogeneity.

    \item[Proof of \subcref{enum:coset_space_T1} and \subcref{enum:coset-space-discrete}]
    First we show that $G/H$ is $T_1$ if and only if $H$ is closed. Note that fibres of $q$ are cosets of $H$, and a coset $gH = \leftmult_g(H)$ is closed if and only if $H$ is. But since $G/H$ carries the quotient topology, $gH$ is closed in $G$ if and only if $\{gH\}$ is closed in $G/H$.
    
    Similarly, $H$ (and hence $gH$) is \emph{open} in $G$ if and only if $\{gH\}$ is open in $G/H$, i.e. if and only if $G/H$ is discrete.
\end{proofsec}
\end{proof}

If a subgroup $N$ of a topological group $G$ is normal, we expect that the usual group structure on the (algebraic) quotient group $G/N$ is compatible with the quotient topology. This is indeed the case:

\begin{theorem}[Topological quotient groups]
    \label{thm:topological-quotient-group}
    If $N$ is a normal subgroup of a topological group $G$, then $G/N$ is a topological group.
\end{theorem}

\begin{proof}
    Let $\mu \colon G \prod G \to G$ and $M \colon G/N \prod G/N \to G/N$ denote multiplication on $G$ and $G/N$ respectively, let $q \colon G \to G/N$ be the quotient map and define a map $Q \colon G \prod G \to G/N \prod G/N$ by $Q = q \prod q$. Then $Q$ is surjective and open since $q$ is. Notice that the diagram
    %
    \begin{equation*}
        \begin{tikzcd}
            G \prod G
                \ar[r, "\mu"]
                \ar[d, "Q", swap]
            & G
                \ar[d, "q"]
            \\
            G/N \prod G/N
                \ar[r, "M", swap]
            & G/N
        \end{tikzcd}
    \end{equation*}
    %
    commutes. If $V \subseteq G/N$ is open, then so is $Q\preim(M\preim(V)) = \mu\preim(q\preim(V))$, so applying $Q$ to both sides we find that $M\preim(V)$ is open. Hence $M$ is continuous.

    Let $I \colon G/N \to G/N$ be the inversion map. Continuity of $I$ is proved similarly by noticing that the diagram
    %
    \begin{equation*}
        \begin{tikzcd}
            G
                \ar[r, "\iota"]
                \ar[d, "q", swap]
            & G
                \ar[d, "q"]
            \\
            G/N
                \ar[r, "I", swap]
            & G/N
        \end{tikzcd}
    \end{equation*}
    %
    commutes, where $\iota \colon G \to G$ is inversion in $G$.
\end{proof}


\begin{proposition}[Factorisation through quotient group]
    Let $\phi \colon G \to H$ be a continuous group homomorphism between topological groups, and let $N$ be a normal subgroup of $G$. If $N \subseteq \ker\phi$, then there exists a unique set function $\tilde{\phi} \colon G/N \to H$ such that the diagram
    %
    \begin{equation*}
        \begin{tikzcd}[column sep=tiny]
            G
                \ar[rr, "\phi"]
                \ar[dr, "q", swap]
            && H
            \\
            & G/N
                \ar[ur, "\tilde{\phi}", swap]
        \end{tikzcd}
    \end{equation*}
    %
    commutes. Furthermore, $\tilde{\phi}$ is a continuous group homomorphism.
\end{proposition}

\begin{proof}
    Existence and uniqueness of $\tilde{\phi}$ follows from the universal property of quotients in the category of sets. Continuity follows from the same property in the category of topological spaces, and $\tilde{\phi}$ is a group homomorphism by the same property in the category of groups.
\end{proof}

We now explore how a topological group $G$ that is \emph{not} $T_0$ can be made so by quotienting out by a particular subgroup. To do this justice we first recall the \emph{$T_0$-identification} of a topological space $X$: The ordering $x \leq y$ defined by $x \in \closure{\{y\}}$ is called the \emph{specialisation preorder}, and it is easy to show that $x \leq y$ is equivalent to $\nhoodfilter{x} \subseteq \nhoodfilter{y}$. This order gives rise to an equivalence relation $\equiv$, and we say that two points $x,y \in X$ are \emph{topologically indistinguishable} if $x \equiv y$.

It is clear that $X$ is $T_0$ if and only if the relation $\equiv$ is trivial, and it is not difficult to show that the quotient space $X/{\equiv}$, called the \emph{$T_0$-identification} or the \emph{Kolmogorov quotient} of $X$, is indeed $T_0$. In fact, $\equiv$ is the most conservative equivalence relation $\sim$ such that $X/{\sim}$ is $T_0$, though we shall not need this fact.

When we apply this construction to the theory of topological groups, we see that the $T_0$-identification can be understood in terms of quotient groups:

\begin{proposition}[$T_0$-identification of groups]
    \label{thm:quotient-group-T0-identification}
    Let $G$ be a topological group. The subgroup $\closure{\{e\}}$ of $G$ is normal. Furthermore, the $\equiv$-equivalence classes are precisely the left cosets of $\closure{\{e\}}$. It follows that the quotient group $G / \closure{\{e\}}$ is just the $T_0$-identification $G/{\equiv}$ of $G$.
\end{proposition}

\begin{proof}
    The subgroup $\closure{\{e\}}$ is the smallest closed subgroup of $G$, hence it is normal since otherwise intersecting it with one of its conjugates yields a strictly smaller closed subgroup.

    It then suffices to show that, for $x,y \in G$, $x \equiv y$ if and only if $x \closure{\{e\}} = y \closure{\{e\}}$. But this is clear since e.g. $x \closure{\{e\}} = \closure{\{x\}}$ by continuity of multiplication on $G$.
\end{proof}


\begin{corollary}
    \label{cor:closure-of-e-trivial-topology}
    The subspace $\closure{\{e\}}$ carries the trivial topology.
\end{corollary}

TODO follows from specialisation preorder.


It is results like the above that lead to the common assumption that topological groups are $T_1$, since if it is not then we just quotient out by $\closure{\{e\}}$. One might justify this by arguing as follows: If a topology on a set $X$ is to respect the set structure, then there must be a difference between two distinct points that is captured by the topology. But this difference must lie in which neighbourhoods each points has; the topology simply carries no further information than this. So if $x \neq y$, then it must be the case that $x \not\equiv y$, i.e. that $X$ is $T_0$.

On the other hand, a \emph{group} structure on a set $G$ certainly respects the set structure, in that different elements may give rise to different actions: just let $G$ act on itself by multiplication.

Thus if the topology and group structure on a topological group are to be truly compatible, then the topology must respect the underlying set structure, i.e. be $T_0$. In the sequel we shall, however, resist assuming that topological groups are $T_0$ as far as possible.

There is more to say about the $T_0$-identification that is not pertinent to this discussion.\footnote{See \href{https://github.com/dnhansen/topology-separation-axioms}{my notes on separation axioms in topology} for more on the $T_0$-identification and related topics.}


% \section{Metrics}

% Next we consider metrics: A pseudo-metric $\rho$ on a group $G$ is called \emph{(translation) invariant} if $\rho(xz,yz) = \rho(x,y)$ for all $x,y,z \in G$.

% Assume that $G$ is pseudo-metrisable by an invariant pseudo-metric $\rho$. Let $H$ be a subgroup of $G$ and define $\rho_H \colon G/H \times G/H \to [0,\infty)$ by
% %
% \begin{equation*}
%     \rho_H(q(x), q(y))
%         = \inf_{h \in H} \rho(y\inv x, h).
% \end{equation*}
% %
% This is well-defined, since if $q(x) = q(x')$ and $q(y) = q(y')$, then $x\inv x' = h_1$ and $y\inv y' = h_2$ for appropriate choices of $h_1,h_2 \in H$, and so
% %
% \begin{equation*}
%     \rho(y\inv x, h)
%         = \rho((y')\inv x', h_2\inv h h_1).
% \end{equation*}
% %
% We show below that $\rho_H$ is a pseudo-metric, and we call it the \emph{quotient metric} on $G/H$.

% We can interpret the quotient metric as follows: We think of the elements of a group as invertible transformations. Since $H$ is the identity in $G/H$, $\inf_{h \in H} \rho(y\inv x, h)$ measures the distance from $y\inv x$ to the closest transformation that \textquote{does nothing}. If this number is small, then $y\inv x$ can be thought of \textquote{doing almost nothing}. That is, first performing the transformation $x$ and then undoing the transformation $y$ we do almost nothing, so $x$ and $y$ must be almost the same transformation.

% This kind of reasoning also shows why we are only interested in invariant metrics: Otherwise $y\inv x$ being close to $e$ would not immediately imply that $x$ and $y$ are close, so a non-invariant metric would not respect the interpretation of group elements as transformations.


% \begin{proposition}[Quotient metric]
%     \label{thm:coset-space-metric}
%     The map $\rho_H$ is a pseudo-metric on $G/H$ compatible with the quotient topology. If $H$ is normal such that $G/H$ is a group, then $\rho_H$ is invariant. Furthermore, $\rho_H$ is a metric if and only if $H$ is closed.
% \end{proposition}

% \newcommand{\calB}{\mathcal{B}}

% \begin{proof}
%     It is clear that $\rho_H$ is positive and symmetric. We prove the triangle inequality: Let $x,y,z \in G$. For any $h, h' \in H$ we find that
%     %
%     \begin{align*}
%         \rho_H(q(x), q(z))
%             &\leq \rho(z\inv x, hh') \\
%             &\leq \rho(z\inv x, z\inv yh') + \rho(z\inv yh', hh') \\
%             &= \rho(y\inv x, h') + \rho(z\inv y, h).
%     \end{align*}
%     %
%     By taking infima over $h,h' \in H$ on the right-hand side we thus get
%     %
%     \begin{equation*}
%         \rho_H(q(x),q(z))
%             \leq \rho_H(q(x),q(y)) + \rho_H(q(y),q(z))
%     \end{equation*}
%     %
%     as desired. It is clear from the definition that $\rho_H$ is invariant if $H$ is normal.

%     Next we show that $\rho_H$ is compatible with the quotient topology on $G/H$. Let $\calB$ be a neighbourhood basis at $x \in G$. We claim that $q(\calB)$ is a neighbourhood basis at $q(x)$. If $U \subseteq G/H$ is a neighbourhood of $q(x)$, then $q\preim(U)$ is a neighbourhood of $x$. Thus there is a basic neighbourhood $B \in \calB$ such that $B \subseteq q\preim(U)$. Since $q$ is open, it follows that $q(B) \subseteq U$ is a neighbourhood of $q(x)$.

%     Now we claim that $\rho(x,y) < r$ if and only if $\rho_H(q(x),q(y)) < r$, for $r > 0$. The latter is true precisely when $\rho(xy\inv, h) < r$ for all $h \in H$, so in particular when $h = e$, i.e. when $\rho(x,y) < r$. Conversely, if $\rho(x,y) < r$ then
%     %
%     \begin{equation*}
%         \rho_H(q(x),q(y))
%             \leq \rho(xy\inv, e)
%             = \rho(x,y)
%             < r.
%     \end{equation*}
%     %
%     Fix $x \in G$ and notice that $\set{B_\rho(x,r)}{r > 0}$ is a local basis at $x$. Since $q(B_\rho(x,r)) = B_{\rho_H}(q(x),r)$ by the argument above, the set $\set{B_{\rho_H}(q(x),r)}{r > 0}$ is a local basis at $q(x)$. Hence $\rho_H$ is compatible with the topology on $G/H$.

%     Finally, recall that a pseudo-metric is a metric if and only if the metric topology is $T_0$. By \cref{thm:topological_group_regular} this is equivalent to it being $T_1$, and \cref{enum:coset_space_T1} implies that this is the case if and only if $H$ is closed.
% \end{proof}

% [Metric identification]


\section{Metrisation}

\newcommand{\field}{\mathbb{K}}

\newcommand{\GL}[1]{\mathrm{GL}(#1)}

If $X$ is a topological space, then we denote by $C(X)$ the space of continuous real-valued functions on $X$. Further denote by $C_b(X)$ the subspace of bounded functions, and equip this space with the supremum norm.

\begin{theorem}[The Birkhoff--Kakutani Theorem]
    Let $G$ be a topological group. Then $G$ is pseudometrisable if and only if it first countable. In this case $G$ is pseudometrisable by a left-invariant pseudometric.
\end{theorem}

\begin{proof}
    The \enquote{only if} direction is obvious, so we prove the \enquote{if} direction.

    Consider the (left) regular representation
    %
    \begin{align*}
        L \colon G &\to \GL{C_b(G)}, \\
        g &\mapsto L_g,
    \end{align*}
    %
    where $L_g f(x) = f(g\inv x)$ for $f \in C_b(G)$ and $x \in G$. The map is easily seen to be well-defined and a representation of $G$. Furthermore, notice that $\norm{L_g f}_{\sup} = \norm{f}_{\sup}$, since acting on $f$ with $L_g$ simply permutes the domain $G$ of $f$, so $f$ and $L_g f$ have the same image.
    
    Fixing a function $f \in C_b(G)$ yields a map $G \to C_b(G)$ given by $g \mapsto L_g f$, and this induces a pseudometric $\rho_f$ on $G$. This is more concretely given by
    %
    \begin{equation*}
        \rho_f(g,h)
            = \norm{L_g f - L_h f}_{\sup}
            = \sup_{x \in G} \abs[\big]{f(g\inv x) - f(h\inv x)}.
    \end{equation*}
    %
    Notice that $\rho_f$ is indeed left-invariant, since
    %
    \begin{align*}
        \rho_f(xg,xh)
            &= \norm{L_{xg} f - L_{xh} f}_{\sup}
             = \norm{L_x(L_g f - L_h f)}_{\sup} \\
            &= \norm{L_g f - L_h f}_{\sup}
             = \rho_f(g,h),
    \end{align*}
    %
    for all $x \in G$.

    Now assume that $G$ is first countable, and let $f$ be the function from \cref{lem:complete-regularity-lemma}. Notice that the map $g \mapsto L_g f$ is continuous by uniform continuity of $f$. It remains to be shown that $\rho_f$ generates the topology on $G$. First notice that every $\rho_f$-ball is open in $G$ since it is the preimage of a ball under $g \mapsto L_g f$. Conversely, let $U$ be open in $G$, let $g \in U$, and let $V_n = \{f > 1 - \tfrac{1}{n}\}$. The $V_n$ constitute a local basis, so there is an $n \in \naturals$ such that $g V_n \subseteq U$. Now let $h \in G$ with $\rho_f(g,h) < 1/n$. Then
    %
    \begin{align*}
        \frac{1}{n}
            &> \rho_f(g,h)
             = \sup_{x \in G} \abs{f(g\inv x) - f(h\inv x)} \\
            &\geq \abs{f(g\inv h) - f(e)}
             \geq 1 - f(g\inv h).
    \end{align*}
    %
    Hence $h \in gV_n$, so $U$ is $\rho_f$-open. This proves the claim.
\end{proof}


\chapter{Topological vector spaces}

\section{Convexity}

\newcommand{\conv}{\operatorname{Conv}}

Below we let $\field$ denote either field of real numbers or the field of complex numbers.

Let $X$ be a $\field$-vector space. For $x,y \in X$ we denote by $[x,y]$ the line segment between $x$ and $y$, i.e. the set $\set{tx + (1-t)y}{t \in [0,1]}$. A subset $C \subseteq X$ is \emph{convex} if $[x,y] \subseteq C$ for all $x,y \in X$, or equivalently if $tC + (1-t)C \subseteq C$ for all $t \in [0,1]$. The intersection of all convex sets in $X$ containing a subset $A \subseteq X$ is called the \emph{convex hull} of $A$ and is denoted $\conv(A)$. This is clearly the smallest convex set containing $A$. Notice that $[x,y] = \conv(\{x,y\})$.

Furthermore, a subset $S$ of $X$ is called \emph{star-shaped at $x \in S$} if $[x,y] \subseteq S$ for all $y \in S$. If there exists an $x \in S$ such that $S$ is star-shaped at $x$, then $S$ is simply called \emph{star-shaped}. Clearly every non-empty convex set is star-shaped.

\begin{lemma}
    Let $X$ and $Y$ be $\field$-vector spaces, let $A, C \subseteq X$ with $C$ convex, and let $T \colon X \to Y$ be a linear map.
    %
    \begin{enumlem}
        \item \label{enum:image-of-convex-set} The image $T(C)$ is convex.
        \item \label{enum:convex-hull-isomorphism} If $T$ is an isomorphism, then
        %
        \begin{equation*}
            T(\conv(A)) = \conv(T(A)).
        \end{equation*}
    \end{enumlem}
\end{lemma}

\begin{proof}
\begin{proofsec}
    \item[Proof of \subcref{enum:image-of-convex-set}]
    This follows easily since
    %
    \begin{equation*}
        t T(C) + (1-t) T(C)
            = T(tC + (1-t)C)
            \subseteq T(C)
    \end{equation*}
    %
    by linearity.

    \item[Proof of \subcref{enum:convex-hull-isomorphism}]
    Clearly $T(A) \subseteq T(\conv(A))$, so by (i) and the minimality of the convex hull we have $\conv(T(A)) \subseteq T(\conv(A))$. Replacing $A$ by $T(A)$ and $T$ by $T\inv$ we get $\conv(A) \subseteq T\inv(\conv(T(A)))$, and applying $T$ to both sides yields the opposite inclusion.
\end{proofsec}
\end{proof}


\section{Definitions and basic properties}

\begin{definition}[Topological vector spaces]
    A \emph{topological vector space} over $\field$ is a tuple $(X, +, \cdot, \calT)$ such that $(X, +, \cdot)$ is a $\field$-vector space, $(X, \calT)$ is a topological space, and both the addition $+ \colon X \prod X \to X$ and the scalar multiplication $\cdot \colon \field \prod X \to X$ are continuous.
\end{definition}
%
Notice that the inversion map $\iota \colon X \to X$ given by $\iota(x) = -x$ can be written $\iota(x) = (-1)x$, hence is continuous. Thus if $(X, +, \cdot, \calT)$ is a topological vector space, then $(X, +, \calT)$ is a topological group.

\begin{definition}
    Let $X$ be a $\field$-vector space, and let $A \subseteq X$. Then $A$ is said to be
    %
    \begin{enumdef}
        \item \emph{balanced} if $\alpha A \subseteq A$ for all $\alpha \in \field$ with $\abs{\alpha} \leq 1$, and
        \item \emph{absorbing} if for every $x \in X$ there exists a $t > 0$ such that $x \in tA$.
    \end{enumdef}
    %
    Assume that $X$ is a topological vector space. Then $A$ is called
    %
    \begin{enumdef}[resume]
        \item \emph{bounded} if for every neighbourhood $U$ of $0$ there exists a $t > 0$ such that $A \subseteq tU$.
    \end{enumdef}
    %
    Furthermore, $X$ is said to be
    %
    \begin{enumdef}[resume]
        \item \emph{locally star-shaped} if it has a basis of star-shaped sets,
        \item \emph{locally convex} if it has a basis of convex sets,
        \item \emph{locally bounded} if each point has a bounded neighbourhood,
        \item an \emph{$F$-space} if its topology is induced by a complete invariant metric, and
        \item a \emph{Fréchet space} if it is a locally convex $F$-space.
    \end{enumdef}
\end{definition}


\begin{remark}
    We collect a series of elementary results concerning the definitions above.
    %
    \begin{enumrem}
        \item \label{rem:transformation-of-balanced-set} If $T \colon X \to Y$ is a linear map between (not necessarily topological) $\field$-vector spaces and $A \subseteq X$ is balanced, then $T(A)$ is also balanced.
        
        \item \label{enum:convex-hull-balanced} The convex hull of a balanced set is balanced: If $\alpha \in \field \setminus \{0\}$, then multiplication by $\alpha$ is a linear isomorphism. If $A$ is balanced, it follows from \cref{enum:convex-hull-isomorphism} that
        %
        \begin{equation*}
            \alpha \conv(A)
                = \conv(\alpha A)
                \subseteq \conv(A),
        \end{equation*}
        %
        so $\conv(A)$ is also balanced.
        
        \item If $A$ is balanced and $\alpha \in \field$ with $\abs{\alpha} = 1$, then both $\alpha A \subseteq A$ and $\alpha\inv A \subseteq A$. The latter implies that $A \subseteq \alpha A$, so $\alpha A = A$. Letting $\alpha = -1$ we get that balanced sets are symmetric. If $\field = \reals$ and $A$ is convex, then $A$ is symmetric iff it is balanced.
        
        \item Absorbing sets automatically contain $0$. So do nonempty balanced sets.

        \item There is another common definition of boundedness that is slightly more complicated than the above but useful in some applications. In \cref{thm:boundedness-equivalent-definitions} we show that these definitions are equivalent.

        Boundedness in the above sense, call it \enquote{$\calT$-boundedness}, does not generally agree with boundedness with respect to a metric $\rho$, call it \enquote{$\rho$-boundedness}: Assume that $X \neq 0$ is metrisable by an invariant metric $\rho$ (e.g. assume that $X$ is normable). Then the metric $\rho' = \rho/(1 + \rho)$ is also invariant and topologically equivalent to $\rho$, and every subset of $X$ is $\rho'$-bounded. However, in \cref{enum:Hausdorff-space-unbounded} we will see that no nontrivial Hausdorff TVS is $\calT$-bounded.

        However, if $\norm{\,\cdot\,}$ is a norm on $X$, then $\calT$-boundedness \emph{does} coincide with $\norm{\,\cdot\,}$-boundedness.
        
        \item $X$ is locally star-shaped iff it has a local basis of star-shaped open sets. Similarly for local convexity, local path-connectedness and local connectedness. In \cref{enum:balanced-local-basis} we prove that all topological vector spaces are locally star-shaped, hence locally (path-)connected.

        \item $X$ is locally bounded iff $0$ has a bounded neighbourhood. Notice that local boundedness is atypical since it does not assume a basis of bounded sets.
    \end{enumrem}
\end{remark}


\begin{lemma}
    Let $X$ be a topological vector space, and let $A \subseteq X$.
    %
    \begin{enumlem}
        \item \label{enum:closure-of-subspace} If $A$ is a subspace, then so is $\closure{A}$.
        
        \item \label{enum:convex-closure-interior} If $A$ is convex, then so are $\closure{A}$ and $\interior{A}$. In particular, the convex hull of an open set is open.
        
        \item \label{enum:balanced-closure-interior} If $A$ is balanced, then so is $\closure{A}$. If $0 \in A$, then $\interior{A}$ is also balanced.

        \item \label{enum:balanced-implies-connected} If $A$ is nonempty and balanced, then it is also star-shaped at $0$, and hence path-connected.
        
        \item \label{enum:bounded-closure-interior} If $A$ is bounded, then so are $\closure{A}$ and $\interior{A}$.
    \end{enumlem}
\end{lemma}

\begin{proof}
\begin{proofsec}
    \item[Proof of \subcref{enum:closure-of-subspace}]
    Assume that $A$ is a subspace, and let $\alpha \in \field$. Then
    %
    \begin{equation*}
        \alpha \closure{A} + \closure{A}
            = \closure{\alpha A} + \closure{A}
            \subseteq \closure{\alpha A + A}
            \subseteq \closure{A}.
    \end{equation*}
    
    \item[Proof of \subcref{enum:convex-closure-interior}]
    Assume that $A$ is convex, and let $t \in [0,1]$. Then
    %
    \begin{equation*}
        t \closure{A} + (1-t) \closure{A}
            = \closure{tA} + \closure{(1-t)A}
            \subseteq \closure{tA + (1-t)A}
            \subseteq \closure{A},
    \end{equation*}
    %
    as desired. Since $\interior{A} \subseteq A$, for $t \in (0,1)$ we have
    %
    \begin{equation*}
        t \interior{A} + (1-t) \interior{A}
            \subseteq A,
    \end{equation*}
    %
    and the set on the left-hand side is open by \cref{enum:product_open_set} (since both $t$ and $1-t$ are nonzero), hence contained in $\interior{A}$. The final claim follows since if $A$ is open, then $\interior{\conv(A)}$ is an open convex set containing $A$.

    \item[Proof of \subcref{enum:balanced-closure-interior}]
    Assume that $A$ is balanced, and let $\alpha \in \field$ with $\abs{A} \leq 1$. Then
    %
    \begin{equation*}
        \alpha \closure{A}
            = \closure{\alpha A}
            \subseteq \closure{A}
    \end{equation*}
    %
    as desired. If $\alpha \neq 0$, then
    %
    \begin{equation*}
        \alpha \interior{A}
            = \interior{(\alpha A)}
            \subseteq \alpha A
            \subseteq A.
    \end{equation*}
    %
    The set of the left-hand side is open, so it is contained in $\interior{A}$. If $A$ contains $0$, then this also holds when $\alpha = 0$.

    \item[Proof of \subcref{enum:balanced-implies-connected}]
    This is obvious if $A$ is empty, so let $x \in A$. For $t \in [0,1]$ we have $tx \in tA \subseteq A$ since $A$ is balanced, so $[0,x] \subseteq A$. Hence $A$ is star-shaped at $0$.
    
    \item[Proof of \subcref{enum:bounded-closure-interior}]
    Assume that $A$ is bounded. Every subset of $A$ is also bounded, so in particular $\interior{A}$ is bounded. Let $U$ be a neighbourhood of $0$. By regularity of $X$ (cf. \cref{thm:topological_group_regular}) there is a neighbourhood $V$ of $0$ such that $\closure{V} \subseteq U$. Since $A$ is bounded there exists a $t > 0$ such that $A \subseteq tV$. We thus have
    %
    \begin{equation*}
        \closure{A}
            \subseteq \closure{tV}
            = t \closure{V}
            \subseteq tU,
    \end{equation*}
    %
    as desired.
\end{proofsec}
\end{proof}


\begin{proposition}[Balanced local basis]
    Let $X$ be a topological vector space.
    %
    \begin{enumprop}
        \item \label{enum:balanced-local-basis} Every open neighbourhood of $0$ contains an open balanced neighbourhood of $0$. In particular, $X$ has a local basis of balanced open sets. Thus $X$ is locally star-shaped, hence locally path-connected.

        \item \label{enum:convex-balanced-local-basis} Every convex open neighbourhood of $0$ contains a convex open balanced neighbourhood of $0$. In particular, if $X$ is locally convex, it has a local basis of convex balanced open sets.
    \end{enumprop}
\end{proposition}

\begin{proof}
\begin{proofsec}
    \item[Proof of \subcref{enum:balanced-local-basis}]
    Let $U$ be an open neighbourhood of $0$. Since scalar multiplication is continuous at $(0,0) \in \field \prod X$, there exists a $\delta > 0$ and an open neighbourhood $V \subseteq X$ of $0$ such that $\ball{0}{\delta} V \subseteq U$. The set $\ball{0}{\delta} V$ is clearly balanced and contains $0$, and since
    %
    \begin{equation*}
        \ball{0}{\delta} V
            = \bigunion_{\alpha \in \ball{0}{\delta}} \alpha V
            = \bigunion_{\alpha \in \pball{0}{\delta}} \alpha V,
    \end{equation*}
    %
    it is a union of open sets and hence itself open. The last claim follows from \cref{enum:balanced-implies-connected}.

    \item[Proof of \subcref{enum:convex-balanced-local-basis}]
    Let $U$ be an open convex neighbourhood of $0$. Part \subcref{enum:balanced-local-basis} furnishes a balanced open neighbourhood $V$ of $0$ contained in $U$. Its convex hull $\conv(V)$ is then balanced by \cref{enum:convex-hull-balanced} and open by \cref{enum:convex-closure-interior}. By minimality it is also contained in $U$, as desired.
\end{proofsec}
\end{proof}


\section{Boundedness}

\begin{corollary}[Alternative characterisation of boundedness]
    \label{thm:boundedness-equivalent-definitions}
    Let $X$ be a topological vector space, and let $A \subseteq X$. Then the following are equivalent:
    %
    \begin{enumcor}
        \item \label{enum:boundedness-definition-restatement} $A$ is bounded, i.e. for every neighbourhood $U$ of $0$ there exists a $t > 0$ such that $A \subseteq tU$.
        \item \label{enum:boundedness-equivalent-definition} For every neighbourhood $U$ of $0$ there exists a $t > 0$ such that $A \subseteq \alpha U$ for all $\alpha \in \field$ with $\abs{\alpha} \geq t$.
    \end{enumcor}
\end{corollary}

\begin{proof}
\begin{proofsec}
    It is obvious that \subcref{enum:boundedness-equivalent-definition} implies \subcref{enum:boundedness-definition-restatement}, so we prove the converse.

    By \cref{enum:balanced-local-basis} $U$ contains a balanced neighbourhood $V$ of $0$. If $A$ is bounded, then there is a $t > 0$ such that $A \subseteq tV$. Let $\alpha \in \field$ with $\abs{\alpha} \geq t$. Then $\alpha V$ is also balanced by \cref{rem:transformation-of-balanced-set}, so since $\abs{t/\alpha} \leq 1$ we have
    %
    \begin{equation*}
        tV
            = \frac{t}{\alpha} \alpha V
            \subseteq \alpha V,
    \end{equation*}
    %
    implying that $A \subseteq \alpha V \subseteq \alpha U$ as desired.
\end{proofsec}
\end{proof}


\begin{corollary}[Hausdorff spaces are unbounded]
    \label{enum:Hausdorff-space-unbounded}
    Let $X$ be a nontrivial Hausdorff topological vector space. Then $X$ is unbounded.
\end{corollary}
%
If $X$ carries the trivial topology, then $X$ is bounded. Hence the Hausdorff assumption cannot simply be dropped.

\begin{proof}
    Let $x \in X$ be nonzero, and let $U$ be a neighbourhood of $0$ not containing $x$. Consider the set $A = \set{nx}{n \in \naturals}$. Assume towards a contradiction that $A$ is bounded, and choose $t > 0$ in accordance with \cref{enum:boundedness-equivalent-definition}. If $n \in \naturals$ with $n \geq t$, then $A \subseteq nU$. But since $x \not\in U$ we have $nx \not\in nU$, a contradiction. Hence $A$ is unbounded, so $X$ itself is unbounded.
\end{proof}


\begin{proposition}
    Let $X$ be a topological vector space, and let $U$ be a neighbourhood of $0$.
    %
    \begin{enumprop}
        \item \label{enum:0-neighbourhood-absorbing} Let $(r_i)_{i \in I}$ be an unbounded net in $\field$. Then
        %
        \begin{equation*}
            X = \bigunion_{i \in I} r_i U.
        \end{equation*}
        %
        In particular, $U$ is absorbing.

        \item \label{enum:compact-implies-bounded} Every compact set $K$ in $X$ is bounded.

        \item \label{enum:bounded-nhood-local-basis} Assume that $U$ is bounded, and let $(\delta_i)_{i \in I}$ be a net in $\field \setminus \{0\}$ with $0$ as a cluster point. Then the family $\set{\delta_i U}{i \in I}$ is a local basis for $X$.
    \end{enumprop}
\end{proposition}

\begin{proof}
\begin{proofsec}
    \item[Proof of \subcref{enum:0-neighbourhood-absorbing}]
    Let $x \in X$. Since scalar multiplication is continuous, the set $V = \set{\alpha \in \field}{\alpha x \in U}$ is a neighbourhood of $0 \in \field$, so there exists an $i \in I$ such that $1/r_i \in V$. That is, $(1/r_i)x \in U$, or $x \in r_i U$. To show that $U$ is absorbing, let $I = \naturals$ and $r_i = i$.

    \item[Proof of \subcref{enum:compact-implies-bounded}]
    To prove that $K$ is bounded, let $W \subseteq U$ be a balanced neighbourhood of $0$. Then \subcref{enum:0-neighbourhood-absorbing} implies that
    %
    \begin{equation*}
        K \subseteq \bigunion_{i \in \naturals} i W,
    \end{equation*}
    %
    and by compactness we have $K \subseteq \bigunion_{i=1}^t i W$ for some $t \in \naturals$. But then $K \subseteq tW$ since $W$ (and hence $tW$ by \cref{rem:transformation-of-balanced-set}) is balanced. It follows that $K \subseteq tU$, so $K$ is bounded.

    \item[Proof of \subcref{enum:bounded-nhood-local-basis}]
    Finally let $V$ be a neighbourhood of $0$. Since $U$ is bounded we may choose $t > 0$ as in \cref{enum:boundedness-equivalent-definition} (with $U$ in place of $A$ and $V$ in place of $U$). There exists an $i \in I$ such that $\delta_i \in \ball{0}{1/t}$, i.e. $\abs{1/\delta_i} > t$. Hence $U \subseteq (1/\delta_i) V$, i.e. $\delta_i U \subseteq V$ as desired.
\end{proofsec}
\end{proof}


\begin{corollary}[Cauchy sequences are bounded]
    \label{thm:Cauchy-sequence-bounded}
    Let $(x_n)_{n \in \naturals}$ be a Cauchy sequence in a topological vector space $X$. Then the set $\set{x_n}{n \in \naturals}$ is bounded. In particular, convergent sequences are bounded.
\end{corollary}
%
More generally, we say that a net $(x_i)_{i \in I}$ is bounded if the set $\set{x_i}{i \in I}$ is bounded. This result, however, does not hold for general nets.

\begin{proof}
    Let $U$ be a neighbourhood of $0$. By \cref{enum:balanced-local-basis} we may assume that $U$ is balanced. There exists a neighbourhood $V$ of $0$ such that $V + V \subseteq U$, by \cref{thm:symmetric_nhood_of_e}. Since $(x_n)$ is Cauchy, there is an $N \in \naturals$ such that $m,n \geq N$ implies that $x_m - x_n \in V$. In particular, $x_n \in x_N + V \subseteq V + V \subseteq U$.

    Now let $n < N$. Then since $U$ is absorbing by \cref{enum:0-neighbourhood-absorbing}, there is a $t_n > 0$ such that $x_n \in t_n U$. Letting $t = \max\{1, t_1, \ldots, t_{N-1}\}$ we thus have $x_n \in tU$ for all $n \in \naturals$ since $U$ is balanced.
\end{proof}


\section{Product and quotient spaces}

\begin{theorem}[Products]
    
\end{theorem}


\begin{proposition}
    \label{prop:vector-space-sum-of-Hausdorff-and-trivial}
    Let $X$ be a topological $\field$-vector space. If $U$ is a complement of $\closure{\{0\}}$ in $X$, then $U$ is Hausdorff and
    %
    \begin{equation*}
        X
            \cong U \oplus \closure{\{0\}}.
    \end{equation*}
    %
    In particular, $X$ is the sum of a Hausdorff space and a trivial space.
\end{proposition}

\begin{proof}
    Since $U$ is a topological vector space, it suffices to show that it is $T_1$. But
    %
    \begin{equation*}
        \closureop_U \{0\}
            = U \intersect \closure{\{0\}}
            = \{0\},
    \end{equation*}
    %
    so this is clear. Next let $S \colon U \oplus \closure{\{0\}} \to V$ be the restriction of the addition map, so it is continuous. This is also easily seen to be a linear isomorphism (using that the sum is direct). If $F$ is a closed neighbourhood of $0$ in $X$, then
    %
    \begin{equation*}
        F
            = \closure{F}
            = \closure{F + \{0\}}
            \supseteq \closure{F} + \closure{\{0\}}
            = F + \closure{\{0\}}
    \end{equation*}
    %
    by TODO ref, so
    %
    \begin{equation*}
        S \bigl( (F \intersect U) \prod \closure{\{0\}} \bigr)
            = (F \intersect U) + \closure{\{0\}}
            = F + \closure{\{0\}}
            = F.
    \end{equation*}
    %
    Since $X$ is regular (cf. TODO ref) it has a local basis of closed sets, so this shows that $S$ is also open, hence a homeomorphism.

    The final claim follows from \cref{cor:closure-of-e-trivial-topology}.
\end{proof}




\begin{theorem}[Topological quotient vector spaces]
    If $M$ is a subspace of a topological $\field$-vector space $X$, then $X/M$ is a topological vector space.
\end{theorem}

\begin{proof}
    \Cref{thm:topological-quotient-group} already shows that $X/M$ is a topological group, so it suffices to show that the scalar multiplication on $X/M$ is continuous. To this end, let $\rho \colon \field \prod X \to X$ be the scalar multiplication on $X$, let $R \colon \field \prod X/M \to X/M$ be the scalar multiplication on $X/M$, let $q \colon X \to X/M$ be the quotient map, and define $Q \colon \field \prod X \to \field \prod X/M$ by $Q = \id_\field \prod q$. Continuity of $R$ then follows by noticing that $Q$ is open and surjective, and that the diagram
    %
    \begin{equation*}
        \begin{tikzcd}
            \field \prod X
                \ar[r, "\rho"]
                \ar[d, "Q", swap]
            & X
                \ar[d, "q"]
            \\
            \field \prod X/M
                \ar[r, "R", swap]
            & X/M
        \end{tikzcd}
    \end{equation*}
    %
    commutes.
\end{proof}


\begin{proposition}[Factorisation through quotient spaces]
    \label{thm:quotient-space-factorisation}
    Let $T \colon X \to Y$ be a continuous linear map between topological vector spaces, and let $M$ be a subspace of $X$. If $M \subseteq \ker T$, then there exists a unique set function $\tilde{T} \colon X/M \to Y$ such that the diagram
    %
    \begin{equation*}
        \begin{tikzcd}[column sep=tiny]
            X
                \ar[rr, "T"]
                \ar[dr, "q", swap]
            && Y
            \\
            & X/M
                \ar[ur, "\tilde{T}", swap]
        \end{tikzcd}
    \end{equation*}
    %
    commutes. Furthermore, $\tilde{T}$ is a continuous linear map.
\end{proposition}

\begin{proof}
    Existence and uniqueness of $\tilde{T}$ follows from the universal property of quotients in the category of sets. Continuity follows from the same property in the category of topological spaces, linearity by the same property in the category of $\field$-vector spaces.
\end{proof}


\section{Continuous linear maps}

A linear map $T \colon X \to Y$ between topological vector spaces is \emph{bounded} if $T(A)$ is bounded for every bounded set $A \subseteq X$. If $X$ and $Y$ are normed spaces, this agrees with the usual definition of boundedness: If there is a $C \geq 0$ such that $\norm{Tx} \leq C\norm{x}$ for all $x \in X$, then $T$ clearly sends bounded sets to bounded sets. Conversely, the closed ball $\cball{0}{1}$ is bounded so $T(\cball{0}{1}) \subseteq \cball{0}{C}$ for some $C \geq 0$, i.e. $\norm{Tx} \leq C$ whenever $\norm{x} \leq 1$. Boundedness with respect to $\norm{\,\cdot\,}$ then follows by linearity.

Continuity and boundedness are equivalent for operators between normed spaces. We begin by exploring the relationship between continuity and boundedness for maps between general topological vector spaces:

\begin{proposition}[Continuity and boundedness]
    \label{thm:continuity-boundedness-equivalent}
    Let $T \colon X \to Y$ be a linear map between topological vector spaces. If $T$ is continuous then it is bounded. The converse also holds if $X$ is first countable.
\end{proposition}

\begin{proof}
    Assume that $T$ is continuous, let $A \subseteq X$ be bounded, and let $V \subseteq Y$ be a neighbourhood of $0$. Letting $U = T\preim(V)$ there exists a $t > 0$ such that $A \subseteq tU$. It follows that $T(A) \subseteq tT(U) = tV$ as desired.

    Conversely, assume that $X$ is first countable and that $T$ is bounded but \emph{not} continuous. Let $(U_n)_{n \in \naturals}$ be a decreasing sequence of sets in $X$ such that $\set{U_n}{n \in \naturals}$ is a local basis. Since $T$ is not continuous, there is a balanced neighbourhood $V$ of $0$ in $Y$ such that $T\preim(V)$ is not a neighbourhood of $0$ in $X$. Hence there exists for every $n \in \naturals$ an $x_n \in \frac{1}{n} U_n$ such that $Tx_n \not\in V$. Then $nx_n \to 0$ as $n \to \infty$, so $(nx_n)_{n \in \naturals}$ is bounded by \cref{thm:Cauchy-sequence-bounded}. Hence the sequence $(nTx_n)_{n \in \naturals}$ is also bounded, so there is a $t > 0$ such that $(nTx_n) \subseteq tV$. Since $V$ is balanced, for $n > t$ we have
    %
    \begin{equation*}
        Tx_n
            \in \frac{t}{n} V
            \subseteq V,
    \end{equation*}
    %
    contradicting the definition of $(x_n)$. Hence $T$ is in fact continuous.
\end{proof}



TODO where? It turns out that linearity and continuity are closely related. In fact, it will turn out that a linear map is continuous if either its domain or codomain is finite-dimensional, at least in the Hausdorff case. Before proving this we note the following result:

\begin{lemma}
    \label{thm:bounded-around-0-implies-continuous}
    Let $T \colon X \to Y$ be a linear map between topological vector spaces. If there is a neighbourhood $U$ of $0$ in $X$ such that $T(U)$ is bounded, then $T$ is continuous.
\end{lemma}

\begin{proof}
    For any neighbourhood $V$ of $0$ in $Y$ there is an $r > 0$ such that $T(rU) = rT(U) \subseteq V$. Since $rU$ is a neighbourhood of $0$, $T$ is continuous at $0$ and thus continuous by \cref{thm:group-homomorphism-continuity}.
\end{proof}




\section{Finite-dimensional spaces}

\newcommand{\eu}{\mathrm{eu}}

We begin by characterising the finite-dimensional Hausdorff spaces.

\begin{lemma}
    Let $Y$ be a topological $\field$-vector space. Then any linear map $T \colon \field^d \to Y$ is continuous.
\end{lemma}

\begin{proof}
    Let $(e_1, \ldots, e_d)$ be the standard basis for $\field^d$. Then
    %
    \begin{equation*}
        Tx = \sum_{i=1}^d \pi_i(x) Te_i
    \end{equation*}
    %
    for $x \in X$, where $\pi_i \colon \field^d \to \field$ is the $i$th projection. Since each $\pi_i$ is continuous, and since addition and scalar multiplication are continuous in $Y$, it follows that $T$ is continuous.
\end{proof}


\begin{theorem}
    Let $V$ be a finite-dimensional $\field$-vector space. There exists a unique topology which makes $V$ into a Hausdorff topological vector space.
\end{theorem}

\begin{proof}
    We first consider the case $V = \field^d$. Consider the identity map
    %
    \begin{align*}
        \id_{\field^d} \colon (\field^d,\calT_{\eu,d}) &\to (\field^d,\calT), \\
        x &\mapsto x.
    \end{align*}
    %
    This is continuous by TODO ref, so $\calT \subseteq \calT_{\eu,d}$.

    To prove the opposite inclusion, notice that it is sufficient to find a neighbourhood of zero $0$ from $\calT$ that is bounded with respect to $\norm{\,\cdot\,}_{\eu,d}$, since we may then translate and scale this neighbourhood to obtain a basis for $\calT_{\eu,d}$. The unit sphere $\sphere^{d-1} = \set{x \in \field^d}{\norm{x}_{\eu,d} = 1}$ is compact in $\calT_{\eu,d}$, so since $\id_{\field^d}$ is continuous it is also compact -- and hence closed -- in $\calT$. There is thus a neighbourhood $U \in \calT$ of $0$ which is disjoint from $\sphere^{d-1}$. Since scalar multiplication is continuous, there is an $\epsilon > 0$ and a neighbourhood $U' \in \calT$ of $0$ such that $(-\epsilon,\epsilon)U' \subseteq U$. For $x \in \field^d \setminus \{0\}$ the vector $x/\norm{x}_{\eu,d}$ lies in $\sphere^{d-1}$. But if $\norm{x}_{\eu,d} > \frac{1}{\epsilon}$, then $1/\norm{x}_{\eu,d} < \epsilon$, and thus $x$ cannot lie in $U'$, since otherwise $x/\norm{x}_{\eu,d}$ would lie in $U$. Hence $U$ is contained in the ball $\overline{B}_{\eu,d}(0,\frac{1}{\epsilon})$ and is thus in particular bounded.

    Finally, if $V$ is any finite-dimensional topological vector space and $T \colon V \to \field^d$ is a linear isomorphism, then this induces a vector space topology on $\field^d$. This topology must equal $\calT_{\eu,d}$, so $T$ is a homeomorphism.
\end{proof}


\begin{corollary}
    Let $V$ be a finite-dimensional $\field$-vector space. All norms on $V$ are then equivalent.
\end{corollary}

\begin{proof}
    TODO
\end{proof}


\begin{corollary}
    \label{thm:finite-dimensional-subspace-closed}
    If $M$ is a finite-dimensional subspace of a Hausdorff topological $\field$-vector space $X$, then $M$ is complete and closed in $X$.
\end{corollary}

\begin{proof}
    Let $d = \dim M$, and let $T \colon \field^d \to M$ be a linear isomorphism, hence a homeomorphism. Then since $\field^d$ is complete as a normed space, hence as a topological vector space by \cref{thm:domain-finite-dimensional-continuity}, $M$ is also complete by \cref{thm:completeness-preserved-by-unimorphism}. But then it is closed by \cref{thm:complete-implies-closed}.
\end{proof}


By \cref{prop:vector-space-sum-of-Hausdorff-and-trivial} it is thus easy to characterise all finite-dimensional topological vector spaces:

\begin{corollary}
    Let $X$ be a finite-dimensional topological $\field$-vector space. Then there is a space $W$ with the trivial topology such that
    %
    \begin{equation*}
        X
            \cong \field^d \oplus W.
    \end{equation*}
\end{corollary}

\begin{proof}
    By \cref{prop:vector-space-sum-of-Hausdorff-and-trivial} $X$ has a Hausdorff subspace $U$ such that $X \cong U \oplus \closure{\{0\}}$, and $U$ is isomorphic to $\field^d$ for some $d$.
\end{proof}









\newcommand{\im}{\operatorname{im}}

\begin{theorem}[Finite-dimensional domain]
    \label{thm:domain-finite-dimensional-continuity}
    Let $T \colon X \to Y$ be a linear map between topological $\field$-vector spaces. If $X$ is Hausdorff and finite-dimensional, then $T$ is continuous. If $Y$ is also Hausdorff and $T$ is injective, then $T$ is a homeomorphism onto its image.
\end{theorem}

\begin{proof}
    Let $q \colon X \to X/\ker T$ be the quotient map, and notice that since $\ker T$ is a finite-dimensional Hausdorff space, it is complete [TODO] and hence closed in $X$.\footnote{Note that we cannot appeal to continuity of $T$ to show that $\ker T$ is closed.} Thus TODO ref implies that $X/\ker T$ is a Hausdorff topological vector space, and since it is finite-dimensional it is isomorphic to $\field^d$ for some $d$. Hence the map $\tilde{T} \colon X/\ker T \to Y$ from TODO ref is continuous by TODO ref. But this implies that $T = \tilde{T} \circ q$ is also continuous.

    If $Y$ is also Hausdorff and $T$ is injective, then $\im T$ is a finite-dimensional Hausdorff space, so the inverse of $T$ must also be continuous. Hence $T$ is a homeomorphism onto its image.
\end{proof}


\begin{theorem}[Finite-dimensional codomain]
    Let $T \colon X \to Y$ be a linear map between topological vector spaces. Assume that $Y$ is finite-dimensional. If $\ker T$ is closed, then $T$ is continuous. The converse also holds if $Y$ is Hausdorff.
\end{theorem}

\begin{proof}
    Let $q \colon X \to X/\ker T$ be the quotient map. Then there is (cf. \cref{thm:quotient-space-factorisation}) a map $\tilde{T} \colon X/\ker T \to Y$ such that $T = \tilde{T} \circ q$, and $\tilde{T}$ is a linear isomorphism onto a subspace of $Y$, so $X/\ker T$ is a finite-dimensional subspace of $X$. If $\ker T$ is closed, then by \cref{enum:coset_space_T1} $X/\ker T$ is a Hausdorff topological vector space, so \cref{thm:domain-finite-dimensional-continuity} implies that $\tilde{T}$ is continuous. Thus $T$ is also continuous.

    Conversely assume that $Y$ is Hausdorff and $T$ is continuous. Then $\{0\}$ is closed in $Y$, so $\ker T = T\preim(\{0\})$ is also closed.
\end{proof}


% \begin{theorem}
%     \label{thm:linear-functional-continuity}
%     Let $\phi \colon X \to \field$ be a nonzero linear functional on a topological $\field$-vector space $X$. Then the following are equivalent:
%     %
%     \begin{enumthm}
%         \item $\phi$ is continuous,
%         \item $\ker\phi$ is closed,
%         \item $\ker\phi$ is not dense in $X$, and
%         \item $\phi$ is bounded in some neighbourhood of $0$.
%     \end{enumthm}
% \end{theorem}

% \begin{proof}
% \begin{proofsec}
%     \item[(i) $\implies$ (ii)]
%     This is clear since $\{0\}$ is closed in $\field$.

%     \item[(ii) $\implies$ (iii)]
%     Since $\ker\phi$ is closed, it suffices to show that $\ker\phi$ is a proper subset of $X$. But this is obvious since $\phi$ is nonzero.

%     \item[(iii) $\implies$ (iv)]
%     There exists an $x \in X$ and a neighbourhood $U$ of $0$ such that $(x + U) \intersect \ker\phi = \emptyset$. By \cref{enum:balanced-local-basis} we may assume that $U$ is balanced, and $\phi(U)$ is then also balanced by \cref{rem:transformation-of-balanced-set}. If $\phi(U)$ is bounded then we are done, so suppose not. Then we must have $\phi(U) = \field$ since the set is unbounded and balanced, so there exists a $y \in U$ such that $\phi(y) = -\phi(x)$. But then $\phi(x+y) = 0$, a contradiction.

%     \item[(iv) $\implies$ (i)]
%     Let $U$ be such a neighbourhood and choose $R > 0$ such that $R \geq \sup_{x \in U} \abs{\phi(x)}$. Let $\epsilon > 0$ and let $U' = (\epsilon/R) U$. Then
%     %
%     \begin{equation*}
%         \phi(U')
%             = \frac{\epsilon}{R} \phi(U)
%             \subseteq \frac{\epsilon}{R} \ball{0}{R}
%             = \ball{0}{\epsilon},
%     \end{equation*}
%     %
%     showing that $\phi$ is continuous at $0$. Continuity everywhere then follows from \cref{thm:group-homomorphism-continuity}.
% \end{proofsec}
% \end{proof}




% \begin{theorem}
%     \label{thm:linear-implies-continuous-finite-dimension}
%     Let $X$ and $Y$ be Hausdorff topological $\field$-vector spaces with $X$ finite-dimensional, and let $T \colon X \to Y$ be a linear map. Then $T$ is continuous, and if $T$ is a linear isomorphism then it is a homeomorphism.
% \end{theorem}
% %
% [TODO: Hausdorff might be necessary? At least the trivial topology is always a vector topology, but does "not T0" imply trivial topology in finite dimension?]

% This implies that there exists exactly one Hausdorff topology on a finite-dimensional vector space $X$ that makes $X$ into a topological vector space. In particular, all norms on $X$ are topologically equivalent; in fact all norms are Lipschitz equivalent, though this does not readily follow from the above.

% \begin{proof}
%     First assume that $X = \field^d$ for some $d \in \naturals$, and that $\field^d$ is equipped with the standard topology.
%     %
%     \begin{proofsec}
%         \item[$T$ is continuous]
%         Let $(e_1, \ldots, e_d)$ be the standard basis for $\field^d$. Then
%         %
%         \begin{equation*}
%             Tx = \sum_{i=1}^d \pi_i(x) Te_i
%         \end{equation*}
%         %
%         for $x \in X$, where $\pi_i \colon \field^d \to \field$ is the $i$th projection. Since each $\pi_i$ is continuous, and since addition and scalar multiplication are continuous in $Y$, it follows that $T$ is continuous.

%         \item[$T\inv$ is continuous]
%         Now assume that $T$ is bijective, i.e. a linear isomorphism. Let $S$ be the unit sphere in $\field^d$. Then $K = T(S)$ is compact (hence closed) and $0 \not\in K$, so $0$ has a balanced neighbourhood $U$ disjoint from $K$. Then $E = T\inv(U)$ is balanced, hence (path-)connected by \cref{enum:balanced-implies-connected}, and disjoint from $S$. Since $E$ contains $0$ it must lie in $\ball{0}{1}$. Thus each coordinate function $\pi_i \circ T\inv$ is bounded in a neighbourhood of $0 \in Y$, hence continuous by \cref{thm:linear-functional-continuity}. Thus $T\inv$ is continuous.
%     \end{proofsec}
%     %
%     This argument of course also shows that any linear isomorphism $Y \to \field^d$ is a homeomorphism.
    
%     For general $X$ let $R \colon X \to \field^d$ be a linear isomorphism, hence a homeomorphism, and notice that $T = (TR\inv)R$. But $TR\inv$ is a linear isomorphism $\field^d \to Y$ so it is also a homeomorphism. Thus $T$ is a homeomorphism.
% \end{proof}

\newcommand{\Span}{\operatorname{span}}

\begin{theorem}[F. Riesz's Theorem]
    Let $X$ be a Hausdorff topological $\field$-vector space. Then $X$ is locally compact if and only if it is finite-dimensional.
\end{theorem}

\begin{proof}
    If $X$ is finite-dimensional of dimension $d$, then $X$ isomorphic to $\field^d$ as a vector space, hence as a topological space by \cref{thm:domain-finite-dimensional-continuity}, and so it is locally compact.

    Conversely, if $X$ is locally compact then $0$ has a precompact open neighbourhood $U$. By \cref{enum:compact-implies-bounded}, $\closure{U}$ (and hence $U$) is also bounded, so the collection $\calU = \set{2^{-n}U}{n \in \naturals}$ is a local basis by \cref{enum:bounded-nhood-local-basis}.

    By compactness of $\closure{U}$ there exists a finite set $\calB \subseteq X$ such that $U \subseteq \closure{U} \subseteq \calB + \frac{1}{2} U$. Hence if $M = \Span\calB$, then $U \subseteq M + \frac{1}{2} U$. Since $M$ is a subspace, it follows that $\frac{1}{2} U \subseteq M + \frac{1}{4} U$. Hence
    %
    \begin{equation*}
        U
            \subseteq M + \tfrac{1}{2} U
            \subseteq M + (M + \tfrac{1}{4} U)
            = M + \tfrac{1}{4} U.
    \end{equation*}
    %
    By induction we thus find that
    %
    \begin{equation*}
        U
            \subseteq \bigintersect_{n \in \naturals} (M + 2^{-n}U)
            = \closure{M}
            = M,
    \end{equation*}
    %
    where the first equality follows from \cref{enum:closure-intersection-of-open-sets} since $\calU$ is a local basis, and the second follows by \cref{thm:finite-dimensional-subspace-closed}.  Since $U$ is absorbing so is $M$, but $M$ is a subspace so $X = M$. Hence $X$ is finite-dimensional.
\end{proof}


% \section{Quotient spaces}

% We begin by reviewing the basic construction and properties of quotients of general \emph{algebraic} vector spaces. So let $X$ be a vector space over a field $k$, and let $S$ be a subspace of $X$. Then $X$ is in particular an abelian group under addition, and $S$ is a subgroup. We may then construct the quotient \emph{group} $X/S$ whose elements are the cosets $x + S$ for $x \in X$. Let $\pi \colon X \to X/S$, so that $\pi(x) = x + S$.

% The equivalence relation defining the group is also compatible with the scalar multiplication on $X$: For if $x,y \in X$ and $x - y \in S$, then for $r \in k$ we have
% %
% \begin{equation*}
%     rx - ry
%         = r(x - y) \in S,
% \end{equation*}
% %
% since $S$ is a subspace. Thus $\pi(rx) = \pi(ry)$, and we may define multiplication by $r \in k$ in $X/S$ by $r \pi(x) = \pi(rx)$. It is routine to check that this makes $X/S$ into a $k$-vector space, called the \emph{quotient space} of $X$ by $S$. It is easy to see that $\pi$ is linear, and that $\ker \pi = S$.

% \newcommand{\setF}{\mathbb{F}}

% We now specialise to topological vector spaces. Let $X$ be a topological vector space over the field $\setF$. If $S$ is a subspace of $X$, then recall that we give $X/S$ the quotient topology coinduced by the quotient map $\pi$, and that $\pi$ is open.


% \begin{theorem}[Topological quotient spaces]
%     Let $X$ be a topological vector space, and let $S$ be a subspace. Then $X/S$ is a topological vector space.
% \end{theorem}

% \begin{proof}
%     By \cref{thm:topological-quotient-group} it suffices to show that scalar multiplication is continuous. Let $x \in X$ and $\alpha \in \setF$, and let $U$ be a neighbourhood of $\pi(\alpha x)$ in $X/S$. Continuity of scalar multiplication in $S$ at $(\alpha, x)$ implies the existence of neighbourhoods $V \subseteq \setF$ and $W \subseteq X$ of $\alpha$ and $x$ respectively, such that $VW \subseteq \pi\preim(U)$. Since $\pi$ is surjective and linear, it follows that $V \pi(W) \subseteq U$, and since $\pi$ is also open $\pi(W)$ is a neighbourhood of $\pi(x)$. Hence scalar multiplication in $X/S$ is continuous.
% \end{proof}
% %
% Notice that the second part of the proof of \cref{thm:topological-quotient-group} is redundant for topological vector spaces. One may thus omit this part if one is only interested in vector spaces and not in groups in their own right.

% We have already considered quotient metrics on arbitrary topological groups. Here is an important example of a quotient metric:

% \newcommand{\calL}{\mathcal{L}}
% \newcommand{\calE}{\mathcal{E}}

% \begin{example}[$\calL^p(\mu)$- and $L^p(\mu)$-spaces]
%     Let $(X, \calE, \mu)$ be a measure space, and consider the Lebesgue space $\calL^p(\mu)$ of $p$-integrable functions for some $p \in (0,\infty)$. This is given the metric topology induced by the pseudo-metric
%     %
%     \begin{equation*}
%         \rho(f,g)
%             = \int_X \abs{f-g}^p \dif\mu.
%     \end{equation*}
%     %
%     Since $\calL^p(\mu)$ is not $T_0$, we wish to consider its $T_0$-identification. By \cref{thm:quotient-group-T0-identification} we can find this by computing the closure $N$ of the zero function. This is easily seen to be precisely the space of functions that are zero $\mu$-a.e. We denote the quotient space by $L^p(\mu) = \calL^p(\mu)/N$. We easily see that the quotient metric is simply given by $\rho_N([f],[g]) = \rho(f,g)$, which is the same metric as in the metric identification.
% \end{example}

% Next we consider norms: Let $\norm{\,\cdot\,}$ be a semi-norm on $X$. We define a map $\norm{\,\cdot\,}_S \colon X/S \to [0, \infty)$ by
% %
% \begin{equation*}
%     \norm{\pi(x)}_S
%         = \inf_{s \in S} \norm{x - s}.
% \end{equation*}

% \begin{proposition}[Quotient norm]
%     The map $\norm{\,\cdot\,}_S$ is a semi-norm on $X/S$. If $\rho$ is the pseudo-metric induced by the semi-norm on $X$ and $\rho_S$ is the corresponding quotient metric on $X/S$, then
%     %
%     \begin{equation*}
%         \norm{\pi(x)}_S = \rho_S(\pi(x), 0).
%     \end{equation*}
%     %
%     In particular, $\rho_S$ is the pseudo-metric induced by $\norm{\,\cdot\,}_S$, and $\norm{\,\cdot\,}_S$ is compatible with the quotient topology. Furthermore, it is a norm if and only if $S$ is closed.
% \end{proposition}

% \begin{proof}
%     It is clear that $\norm{\,\cdot\,}_S$ is positive and absolutely homogeneous. We next prove the triangle inequality:
    
%     Let $\rho$ be the metric induced by the semi-norm on $X$, i.e. $\rho(y,z) = \norm{y - z}$ for $y,z \in X$. Notice that since $\rho$ is invariant, for $x \in X$,
%     %
%     \begin{align*}
%         \norm{\pi(x)}_S
%             &= \inf_{s \in S} \norm{x - s}
%              = \inf_{s \in S} \rho(x - s, 0) \\
%             &= \inf_{s \in S} \rho(x - 0, s)
%              = \rho_S(\pi(x), 0).
%     \end{align*}
%     %
%     If also $y \in X$, then it follows that
%     %
%     \begin{align*}
%         \norm{\pi(x) + \pi(y)}_S
%             &= \rho_S(\pi(x) + \pi(y), 0)
%              = \rho_S(\pi(x), \pi(-y)) \\
%             &\leq \rho_S(\pi(x), 0) + \rho_S(0, \pi(-y))
%              = \norm{\pi(x)}_S + \norm{\pi(y)}_S,
%     \end{align*}
%     %
%     so $\norm{\,\cdot\,}_S$ is a semi-norm. The rest of the claims follow by \cref{thm:coset-space-metric}.
% \end{proof}


\chapter{Locally convex spaces}

\section{Seminorm topologies}

Let $X$ be a $\field$-vector space. A \emph{seminorm} on $X$ is a sublinear, absolutely homogeneous map $X \to \reals$. A collection $\calP$ of seminorms on $X$ is said to \emph{separate points} or is \emph{separating} if there for every $x \in X \setminus \{0\}$ is a $p \in \calP$ such that $p(x) \neq 0$. Or equivalently if, for $x,y \in X$, $x \neq y$ implies the existence of a $p \in \calP$ with $p(x-y) \neq 0$.

A single seminorm gives rise to a topology in the usual. But often the topology on a vector space is generated by a family of many seminorms:

\begin{theorem}[Generating seminorm topologies]
    \label{thm:seminorm-topology}
    Let $X$ be a $\field$-vector space, and let $\calP$ be a family of seminorms on $X$. Let $\calT$ denote the topology induced by $\calP$, i.e. generated by all open balls $\ball[p]{x}{\epsilon}$ for $p \in \calP$, $x \in X$, and $\epsilon > 0$.

    \begin{enumthm}
        \item \label{enum:LCS-nhood-basis} For each $x \in X$, the finite intersections of the sets $\ball[p]{x}{\epsilon}$, for $p \in \calP$ and $\epsilon > 0$, form a neighbourhood basis at $x$.

        \item \label{enum:LCS-net-convergence} If $(x_i)_{i \in I}$ is a net in $X$ and $x \in X$, then $x_i \to x$ iff $p(x_i - x) \to 0$ for all $p \in \calP$.

        \item \label{enum:seminorm-topology-is-LCS} $(X,\calT)$ is a locally convex topological vector space.

        \item \label{enum:seminorm-topology-separating} $\calT$ is $T_0$ (and hence regular) if and only if $\calP$ separates points in $X$.
    \end{enumthm}
\end{theorem}

\begin{proof}
\begin{proofsec}
    \item[Proof of \subcref{enum:LCS-nhood-basis}]
    Let $U$ be a neighbourhood of $x$. By definition of $\calT$,
    %
    \begin{equation*}
        x
            \in \bigintersect_{k=1}^n \ball[p_k]{x_k}{\epsilon_k}
            \subseteq U
    \end{equation*}
    %
    for appropriate $x_k \in X$, $p_k \in \calP$ and $\epsilon_k > 0$. Letting $\delta_k = \epsilon_k - p_k(x - x_k)$ the triangle inequality implies that
    %
    \begin{equation*}
        x
            \in \bigintersect_{k=1}^n \ball[p_k]{x}{\delta_k}
            \subseteq \bigintersect_{k=1}^n \ball[p_k]{x_k}{\epsilon_k},
    \end{equation*}
    %
    proving the claim.

    \item[Proof of \subcref{enum:LCS-net-convergence}]
    This is obvious from \subcref{enum:LCS-nhood-basis}.

    \item[Proof of \subcref{enum:seminorm-topology-is-LCS}]
    Let $(x_i)_{i \in I}$ and $(y_i)_{i \in I}$ be nets in $X$, and let $(\alpha_i)_{i \in I}$ be a net in $\field$, and assume that $x_i \to x$, $y_i \to y$ and $\alpha_i \to \alpha$. Then eventually $\abs{\alpha_i} \leq \abs{\alpha} + 1$, so eventually
    %
    \begin{align*}
        p\bigl( (\alpha_i x_i + y_i) - (\alpha x + y) \bigr)
            &= p\bigl( (\alpha_i x_i - \alpha_i x) + (\alpha_i x - \alpha x) + (y_i - y) \bigr) \\
            &\leq p(\alpha_i x_i - \alpha_i x) + p(\alpha_i x - \alpha x) + p(y_i - y) \\
            &\leq (\abs{\alpha} + 1) p(x_i - x) + \abs{\alpha_i - \alpha} p(x) + p(y_i - y),
    \end{align*}
    %
    which goes to zero. Hence the vector operations are continuous. Furthermore, the usual proof that balls are convex also works for balls defined by seminorms, so the balls $\ball[p]{x}{\epsilon}$ are convex. Hence (finite) intersections of balls are convex, so $\calT$ has a basis of convex sets.

    \item[Proof of \subcref{enum:seminorm-topology-separating}]
    Let $x,y \in X$ with $x \neq y$. By \subcref{enum:LCS-nhood-basis}, $x$ has a neighbourhood not containing $y$ if and only if there is a ball $\ball[p]{x}{\epsilon}$ for some $p \in \calP$ and $\epsilon > 0$ that does not contain $y$. But $y \not\in \ball[p]{x}{\epsilon}$ if and only if $p(x-y) > 0$.
\end{proofsec}
\end{proof}

\newcommand{\ev}{\mathrm{ev}}

Next we consider a vector space $X$ that has already been equipped with a vector space topology, and we try to equip $X$ with a family of seminorms. Of course such a family can only induce the topology on $X$ if $X$ is locally convex, but the constructions below will be relevant in a more general setting (in particular in the proof of \cref{thm:Hahn-Banach-separation}).

To each $A \subseteq X$ we associate the \emph{Minkowski functional} $\mu_A$ of $A$, given by
%
\begin{equation*}
    \mu_A(x)
        = \inf \set{t > 0}{t\inv x \in A}
\end{equation*}
%
for $x \in X$. We shall only be interested in $\mu_A$ in the case where $A$ is absorbing (and in fact also convex). In this case $A$ is nonempty (since it contains $0$) and we have $\mu_A(x) < \infty$ for all $x \in X$.

\begin{lemma}
    Let $X$ be a topological vector space, and let $A \subseteq X$ be a convex, absorbing set. For all $x, y \in X$ we then have
    %
    \begin{enumlem}
        \item \label{enum:Minkowski-functional-sublinear} $\mu_A(x + y) \leq \mu_A(x) + \mu_A(y)$, and
        
        \item \label{enum:Minkowski-functional-homogeneous} $\mu_A(\alpha x) = \mu_{\alpha\inv A}(x)$ for $\alpha \in \field$, where $0\inv A \defn X$. If $t \geq 0$, then $\mu_A(tx) = t\mu_A(x)$.
    \end{enumlem}
    %
    Assume furthermore that $A$ is balanced.
    %
    \begin{enumlem}[resume]
        \item \label{enum:Minkowski-functional-seminorm} $\mu_A$ is a seminorm.
        
        \item \label{enum:Minkowski-functional-balls} If $A$ is open, $x \in X$ and $r > 0$, then
        %
        \begin{equation*}
            \ball[\mu_A]{x}{r}
                = \set{y \in X}{\mu_A(y-x) < r}
                = x + rA.
        \end{equation*}
    \end{enumlem}
\end{lemma}

\begin{proof}
\begin{proofsec}
    \item[Proof of \subcref{enum:Minkowski-functional-sublinear}]
    Let $\epsilon > 0$, and put $s = \mu_A(x) + \epsilon$ and $t = \mu_A(y) + \epsilon$. Since $A$ is convex and contains $0$, we have $s\inv x, t\inv y \in A$. But then
    %
    \begin{equation*}
        \frac{x+y}{s+t}
            = \frac{t}{s+t} \frac{x}{s} + \frac{s}{s+t} \frac{y}{t}
            \in A
    \end{equation*}
    %
    by convexity, which implies that
    %
    \begin{equation*}
        \mu_A(x + y)
            \leq s + t
            = \mu_A(x) + \mu_A(y) + 2\epsilon.
    \end{equation*}
    %
    Since $\epsilon$ was arbitrary, the claim follows.

    \item[Proof of \subcref{enum:Minkowski-functional-homogeneous}]
    For $\alpha \neq 0$ we have
    %
    \begin{equation*}
        \mu_A(\alpha x)
            = \inf \set{s > 0}{s\inv \alpha x \in A}
            = \inf \set{s > 0}{s\inv x \in \alpha\inv A},
    \end{equation*}
    %
    and if $\alpha = 0$ then since $A$ contains $0$ we have
    %
    \begin{equation*}
        \mu_A(0x) = 0 = \mu_{0\inv A}(x),
    \end{equation*}
    %
    with the convention that $0\inv A = X$. For $t > 0$ we have
    %
    \begin{equation*}
        \mu_A(t x)
            = \inf \set{s > 0}{s\inv t x \in A}
            = \inf \set{tr}{r > 0, r\inv x \in A}
            = t \mu_A(x),
    \end{equation*}
    %
    where we have used the substitution $r = s/t$.
    
    \item[Proof of \subcref{enum:Minkowski-functional-seminorm}]
    Let $\alpha \in \field$ and write $\alpha = \abs{\alpha} u$ with $\abs{u} = 1$. Then
    %
    \begin{align*}
        \mu_A(\alpha x)
            = \mu_A(\abs{\alpha}ux)
            = \abs{\alpha} \mu_{u\inv A}(x)
            = \abs{\alpha} \mu_A(x),
    \end{align*}
    %
    where we use that $u\inv A = A$ since $A$ is balanced.
    
    \item[Proof of \subcref{enum:Minkowski-functional-balls}]
    First assume that $r = 1$ and $x = 0$. If $y \in A$, then there exists a $t \in (0,1)$ such that $t\inv y \in A$ since $A$ is open. Hence $\mu_A(y) < 1$. If instead $y \not\in A$, then if $t\inv y \in A$ for some $t > 0$, then we must have $t \geq 1$ since $A$ is balanced.
    
    For general $r > 0$ notice that
    %
    \begin{equation*}
        rA
            = \set{y \in X}{\mu_{rA}(y) < 1}
            = \set{y \in X}{r\inv \mu_A(y) < 1}
            = \set{y \in X}{\mu_A(y) < r}
    \end{equation*}
    %
    by \subcref{enum:Minkowski-functional-homogeneous}. The claim for general $x \in X$ is obvious.
\end{proofsec}
\end{proof}


% \begin{theorem}
%     \label{thm:locally-convex-Minkowski-functionals}
%     Let $X$ be a locally convex space, and let $\calB$ be a local basis in $X$ of convex, balanced sets. For each $B \in \calB$ the Minkowski functional $\mu_B$ is continuous. If $X$ is Hausdorff, then $\set{\mu_B}{B \in \calB}$ separates points in $X$.
% \end{theorem}
% %
% The existence of such a local basis is guaranteed by \cref{enum:convex-balanced-local-basis}. [TODO: Move continuity, it might be relevant in a non-locally convex space.]

% \begin{proof}
%     \cref{enum:0-neighbourhood-absorbing} shows that each $B$ is absorbing, so we may consider the Minkowski functional $\mu_B$. \cref{enum:Minkowski-functional-seminorm} shows that each $\mu_B$ is a seminorm. Let $\epsilon > 0$. For $x,y \in X$ with $x-y \in \epsilon B$ it follows from [TODO Lemma] and the reverse triangle inequality that
%     %
%     \begin{equation*}
%         \abs{ \mu_B(x) - \mu_B(y) }
%             \leq \mu_B(x - y)
%             < \epsilon.
%     \end{equation*}
%     %
%     Hence $\mu_B$ is continuous.

%     Now assume that $X$ is Hausdorff. If $x \in X \setminus \{0\}$, then there is a $B \in \calB$ with $x \not\in B$. Then $\mu_B(x) \geq 1$ by \cref{enum:Minkowski-functional-balls}, so the family $\set{\mu_B}{B \in \calB}$ separates points in $X$.
% \end{proof}


\begin{remark}
    Let $(X,\calT)$ be a locally convex topological vector space, and let $\calB$ be a local basis of convex balanced open sets, in accordance with \cref{enum:convex-balanced-local-basis}. By \cref{thm:seminorm-topology}, the corresponding family $\calM = \set{\mu_B}{B \in \calB}$ of Minkowski functionals generates a vector space topology $\calT'$ on $X$. We claim that $\calT = \calT'$.

    The inclusion $\calT' \subseteq \calT$ follows since each Minkowski functional is $\calT$-continuous (since seminorms are continuous), so the sets $\ball[\mu]{0}{\epsilon} = \mu\preim(\ball{0}{\epsilon})$ are $\calT$-open for all $\mu \in \calM$. Hence the sets $\ball[\mu]{x}{\epsilon} = x + \ball[\mu]{0}{\epsilon}$ for general $x \in X$ are also $\calT$-open by homogeneity.

    Conversely, for $B \in \calB$ we have $B = \ball[\mu_B]{0}{1}$ by \cref{enum:Minkowski-functional-balls}, so $B \in \calT'$. Hence $\calT \subseteq \calT'$ since the latter is a vector space topology, and the former is generated by translates of sets in $\calB$.
\end{remark}
%
This remark in particular implies the following result:
%
\begin{theorem}
    \label{thm:locally-convex-iff-seminorms}
    A topological vector space $X$ is locally convex if and only if its topology is generated by a family of seminorms.
\end{theorem}


\begin{theorem}
    A topological vector space $X$ is seminormable if and only if $0$ has a convex bounded neighbourhood.
\end{theorem} % [TODO: Where does this result belong?]

\begin{proof}
    The \enquote{only if} part if obvious, so we prove the converse.

    Let $U$ be a convex bounded neighbourhood of $0$. By \cref{enum:convex-balanced-local-basis} we may assume that $U$ is also balanced, so \cref{enum:Minkowski-functional-seminorm} implies that the Minkowski functional $\mu_U$ is a seminorm. It suffices to show that $\mu_U$ generates the topology on $X$.

    By \cref{enum:bounded-nhood-local-basis}, the collection $\set{rU}{r > 0}$ is a local basis. But \cref{enum:Minkowski-functional-balls} says that $rU = \ball[\mu_U]{0}{r}$, so the topology on $X$ coincides with the $\mu_U$-topology.
\end{proof}


\section{The Hahn--Banach theorems}

[TODO find a better place for this stuff about seminorms?]

Note that if $p$ and $q$ are seminorms on a vector space $X$, then $p \leq q$ if and only if $\ball[q]{0}{1} \subseteq \ball[p]{0}{1}$. The \enquote{only if} part is obvious, and the \enquote{if} part follows [TODO?]


\begin{proposition}[Continuity of seminorms]
    Let $X$ be a topological vector space, and let $p$ be a seminorm on $X$. Then the following are equivalent:
    %
    \begin{enumprop}
        \item \label{enum:seminorm-uniformly-continuous} $p$ is uniformly continuous.
        \item \label{enum:seminorm-open-ball} $\ball[p]{0}{1}$ is open in $X$.
        \item \label{enum:seminorm-closed-ball} $\cball[p]{0}{1}$ is a neighbourhood of $X$.
        \item \label{enum:seminorm-continuous-at-0} $p$ is continuous at $0$.
        \item \label{enum:seminorm-continuity-domination} There is a continuous seminorm $q$ on $X$ such that $p \leq q$.
    \end{enumprop}
\end{proposition}

\begin{proof}
    We first establish the equivalence of the first four properties, and then prove equivalence with the final property. Clearly \subcref{enum:seminorm-uniformly-continuous} implies \subcref{enum:seminorm-open-ball}, and \subcref{enum:seminorm-open-ball} implies \subcref{enum:seminorm-closed-ball}.
    %
    \begin{proofsec}
        \item[\subcref{enum:seminorm-closed-ball} $\implies$ \subcref{enum:seminorm-continuous-at-0}]
        If $\cball[p]{0}{1}$ is a neighbourhood of $0$, then so is $\epsilon \cball[p]{0}{1} = \cball[p]{0}{\epsilon}$ for all $\epsilon > 0$. If $(x_i)_{i \in I}$ is a net in $X$ converging to $0$, then eventually $x_i \in \ball[p]{0}{\epsilon}$, so eventually $p(x_i) < \epsilon$. It follows that $p(x_i) \to 0$, so $p$ is continuous at $0$.

        \item[\subcref{enum:seminorm-continuous-at-0} $\implies$ \subcref{enum:seminorm-uniformly-continuous}]
        Let $\epsilon > 0$. By continuity of $p$ at $0$ there is a neighbourhood $U$ of $0$ in $X$ such that $p(U) \subseteq [0,\epsilon)$. By \cref{thm:symmetric_nhood_of_e} there is a symmetric neighbourhood $V$ of $0$ with $V + V \subseteq U$. If $x,y \in V$ then $x-y \in U$, implying that
        %
        \begin{equation*}
            \abs{p(x) - p(y)}
                \leq p(x-y)
                < \epsilon,
        \end{equation*}
        %
        so $p$ is uniformly continuous.

        \item[\subcref{enum:seminorm-uniformly-continuous} $\implies$ \subcref{enum:seminorm-continuity-domination}]
        If $p$ is (uniformly) continuous, then $q = p$ works.

        \item[\subcref{enum:seminorm-continuity-domination} $\implies$ \subcref{enum:seminorm-closed-ball}]
        Note that $p \leq q$ if and only if $\cball[q]{0}{1} \subseteq \cball[p]{0}{1}$. Hence if $q$ is continuous, then $\cball[q]{0}{1}$ is a neighbourhood of $0$ and so is $\cball[p]{0}{1}$.
    \end{proofsec}
\end{proof}


\begin{proposition}
    \label{thm:continuity-by-seminorm-domination}
    Let $X$ and $Y$ be topological $\field$-vector spaces, and assume that $Y$ is locally convex. A linear map $T \colon X \to Y$ is continuous if and only if for each continuous seminorm $q$ on $Y$ there exists a continuous seminorm $p$ on $X$ such that $q \circ T \leq p$.

    In particular, a linear functional $f \colon X \to \field$ is continuous if and only if there is a continuous seminorm $p$ on $X$ such that $\abs{f} \leq p$.
\end{proposition}

\begin{proof}
    If $T$ is continuous and $q$ is a continuous seminorm on $Y$, then $p = q \circ T$ works.

    Conversely, let $q$ be a continuous seminorm on $Y$ and $p$ a continuous seminorm on $X$ such that $q \circ T \leq p$. Since $q \circ T$ itself is a seminorm, \cref{enum:seminorm-continuity-domination} implies that $q \circ T$ is continuous. Now let $(x_i)_{i \in I}$ be a net in $X$ converging to $0$. Then $p(x_i) \to 0$ by continuity, so also $q(Tx_i) \to 0$. Thus $Tx_i \to 0$ by \cref{enum:LCS-net-convergence}, so $T$ is continuous at $0$. Hence it is continuous by \cref{thm:group-homomorphism-continuity}.

    The final claim follows since $\abs{\,\cdot\,}$ is the only seminorm on $\field$ up to multiplication by a non-negative number.
\end{proof}


\begin{lemma}
    \label{thm:real-complex-functionals}
    Let $X$ be a complex vector space over. If $f$ is a complex linear functional on $X$ and $u = \Re f$, then $u$ is a real linear functional, and $f(x) = u(x) - \iu u(\iu x)$ for all $x \in X$. Conversely,  if $u$ is a real linear functional on $X$ and $f \colon X \to \complex$ is defined by $f(x) = u(x) - \iu u(\iu x)$, then $f$ is complex linear. If $X$ is normed, then $\norm{u} = \norm{f}$.
\end{lemma}

\begin{proof}
    See \textcite[Proposition~5.5]{folland2007}.
\end{proof}


\begin{theorem}[The Hahn--Banach Dominated Extension Theorem]
    \label{thm:Hahn-Banach-extension}
    \begin{enumthm}
        \item \label{enum:Hahn-Banach-extension-real} Let $X$ be a real vector space, $p$ a sublinear functional on $X$, $M$ a subspace of $X$, and $f$ a linear functional on $M$ with $f(x) \leq p(x)$ for all $x \in M$. Then there exists a linear functional $F$ on $X$ such that $F(x) < p(x)$ for all $x \in X$ and $F|_M = f$.
        
        \item Let $X$ be a complex vector space, $p$ a seminorm on $X$, $M$ a subspace of $X$, and $f$ a complex linear functional on $M$ with $\abs{f(x)} \leq p(x)$ for all $x \in M$. Then there exists a complex linear functional $F$ on $X$ such that $\abs{F(x)} < p(x)$ for all $x \in X$ and $F|_M = f$.
    \end{enumthm}
\end{theorem}

\begin{proof}
    See \textcite[Theorem~5.6]{folland2007}.
\end{proof}


\begin{corollary}[Continuous extensions]
    Let $X$ be a locally convex topological $\field$-vector space, let $M$ be a subspace of $X$, and let $f \colon M \to \field$ be a continuous linear functional on $M$. Then $f$ has a continuous extension $\phi \colon X \to \field$.
\end{corollary}

\begin{proof}
    Clearly $M$ is locally convex in the subspace topology. Hence if $\calP$ is the collection of continuous seminorms on $X$, the topology on $M$ is generated by the family $\calP|_M = \set{p_M}{p \in \calP}$ of continuous seminorms on $X$ by \cref{thm:locally-convex-iff-seminorms}. \Cref{thm:continuity-by-seminorm-domination} then yields a $p \in \calP$ such that $\abs{f} \leq p$ on $M$, so \cref{thm:Hahn-Banach-extension} implies that $f$ has a linear extension $\phi \colon X \to \field$ with $\abs{\phi} \leq p$. Another application of \cref{thm:continuity-by-seminorm-domination} then yields continuity of $\phi$.
\end{proof}


\begin{lemma}
    \label{thm:functionals-are-open}
    Let $X$ be a topological $\field$-vector space, and let $\phi \colon X \to \field$ be a nonzero linear functional. Then $\phi$ is open.
\end{lemma}

\begin{proof}
    Since $\phi$ is nonzero, there is an $x_0 \in X$ such that $\phi(x_0) = 1$. Let $U \subseteq X$ be an open set, and let $y \in \phi(U)$. Then there is some $x \in U$ such that $\phi(x) = y$. By continuity of scalar multiplication there is a $\delta > 0$ such that $\abs{r} < \delta$ implies that $x + rx_0 \in U$. But then
    %
    \begin{equation*}
        y + r
            = \phi(x + rx_0)
            \in \phi(U)
    \end{equation*}
    %
    for $\abs{r} < \delta$. Thus $\phi(U)$ is open as desired.
\end{proof}


\begin{theorem}[The Hahn--Banach Separation Theorem]
    \label{thm:Hahn-Banach-separation}
    Let $X$ be a topological $\field$-vector space, and let $A,B \subseteq X$ be disjoint, nonempty, convex sets.
    %
    \begin{enumthm}
        \item \label{enum:Hahn-Banach-separation-open} If $A$ is open, then there is a $\phi \in X^*$ and an $\alpha \in \reals$ such that
        %
        \begin{equation*}
            \Re \phi(a)
                < \alpha
                \leq \Re \phi(b)
        \end{equation*}
        %
        for all $a \in A$ and $b \in B$.

        \item \label{enum:Hahn-Banach-separation-compact-closed} If $X$ is locally convex, $A$ is compact, and $B$ is closed, then there exist $\phi \in X^*$ and $\alpha,\beta \in \reals$ such that
        %
        \begin{equation*}
            \Re \phi(a)
                < \alpha
                < \beta
                < \Re \phi(b)
        \end{equation*}
        %
        for all $a \in A$ and $b \in B$.
    \end{enumthm}
\end{theorem}

\begin{proof}
\begin{proofsec}
    \item[Proof of \subcref{enum:Hahn-Banach-separation-open}]
    First assume that $\field = \reals$. Choose points $a_0 \in A$ and $b_0 \in B$, and then let $x_0 = b_0 - a_0$ and $C = A - B + x_0$. Since $A$ is open, \cref{enum:product_open_set} implies that $C$ is an open neighbourhood of $0$. Notice also that $C$ is convex since both $A$ and $B$ are convex.
    
    Consider the Minkowski functional $\mu_C$ of $C$, and notice that $\mu_C(x_0) \geq 1$ by \cref{enum:Minkowski-functional-balls} since $x_0 \not\in C$. Define a linear functional $f \colon \reals x_0 \to \reals$ by $f(tx_0) = t$. For $t \geq 0$ we have
    %
    \begin{equation*}
        f(tx_0)
            = t
            \leq t\mu_C(x_0)
            = \mu_C(tx_0),
    \end{equation*}
    %
    and if $t < 0$ then $f(tx_0) < 0 \leq \mu_C(tx_0)$. Thus $f \leq \mu_C$ on $\reals x_0$, so \cref{enum:Hahn-Banach-extension-real} yields a linear extension $\phi \colon X \to \reals$ of $f$ with $\phi \leq \mu_C$ on $X$. In particular we have $\phi \leq 1$ on $C$, so $\phi \geq -1$ on $-C$ by linearity, which implies that $\abs{\phi} \leq 1$ on the neighbourhood $C \intersect (-C)$ of $0$. It thus follows from \cref{thm:bounded-around-0-implies-continuous} that $\phi$ is continuous.

    Now consider $a \in A$ and $b \in B$, and notice that
    %
    \begin{equation*}
        \phi(a) - \phi(b) + 1
            = \phi(a - b + x_0)
            \leq \mu_C(a - b + x_0)
            < 1,
    \end{equation*}
    %
    since $a - b + x_0 \in C$. Thus $\phi(a) < \phi(b)$ for all $a \in A$ and $b \in B$. It follows that $\alpha \defn \sup \phi(A) \leq \inf \phi(B)$. But $\phi(A)$ is open by \cref{thm:functionals-are-open}, so $\alpha \not\in \phi(A)$. This proves the claim in the case $\field = \reals$.

    Now assume that $\field = \complex$. Then the above yields a continuous real-linear functional $u \colon X \to \reals$ and an $\alpha \in \reals$ such that
    %
    \begin{equation*}
        u(a)
            < \alpha
            \leq u(b)
    \end{equation*}
    %
    for all $a \in A$ and $b \in B$. Define $\phi \colon X \to \complex$ by $\phi(x) = u(x) - \iu u(\iu x)$, so that $u = \Re \phi$. Then $\phi$ is continuous, and \cref{thm:real-complex-functionals} implies that it is complex linear, so $\phi \in X^*$ as desired.

    \item[Proof of \subcref{enum:Hahn-Banach-separation-compact-closed}]
    By \cref{thm:topological_group_regular} there is an open neighbourhood $U$ of $0$ such that $A + U$ and $B$ are disjoint. By \cref{enum:convex-balanced-local-basis} we may assume that $U$ is convex; since $A$ is also convex, so is their sum $A + U$. Part \subcref{enum:Hahn-Banach-separation-open} then yields a $\phi \in X^*$ such that $\Re\phi(A+U)$ and $\Re\phi(B)$ are disjoint subsets of $\reals$. Notice that $\Re\phi(A+U)$ is open by \cref{thm:functionals-are-open} (since $\Re\phi \colon X \to \reals$ is a linear functional), and $\Re\phi(A)$ is a compact subset of $\Re\phi(A+U)$. Hence
    %
    \begin{equation*}
        \sup \Re\phi(A)
            < \sup \Re\phi(A+U)
            \leq \inf \Re\phi(B),
    \end{equation*}
    %
    so the existence of $\alpha, \beta \in \reals$ as in the statement of the theorem is obvious.
\end{proofsec}
\end{proof}


\begin{corollary}
    Let $M$ be a closed subspace of a locally convex topological $\field$-vector space $X$, and let $x_0 \in X \setminus M$. Then there exists a $\phi \in X^*$ with $\phi(x_0) \neq 0$ that vanishes on $M$.
\end{corollary}

\begin{proof}
    \cref{enum:Hahn-Banach-separation-compact-closed} yields a $\phi \in X^*$ such that $\Re \phi(x_0) < \Re \phi(m)$ for all $m \in M$. Since $0 \in M$ we must have $\phi(x_0) \neq 0$. Furthermore, $\phi(M)$ is a proper subspace of $\field$, so $\phi(M) = \{0\}$.
\end{proof}

\begin{corollary}
    \label{thm:X*-separating-points-Hausdorff}
    Let $X$ be a locally convex topological vector space. Then $X$ is Hausdorff if and only if $X^*$ separates points in $X$.
\end{corollary}

\begin{proof}
    Let $x,y \in X$ with $x \neq y$. If $X$ is Hausdorff then singletons are closed, so \cref{enum:Hahn-Banach-separation-compact-closed} yields a $\phi \in X^*$ with $\phi(x) \neq \phi(y)$.

    Conversely, if $X^*$ separates points then there exists a $\phi \in X^*$ with $\phi(x) \neq \phi(y)$. If $U$ and $V$ are disjoint open neighbourhoods of $\phi(x)$ and $\phi(y)$ respectively, then $\phi\preim(U)$ and $\phi\preim(V)$ are disjoint open neighbourhoods of $x$ and $y$ in $X$.
\end{proof}


\chapter{A survey of topologies}

\section{Topologies induced by linear maps}

Let $X$ be a $\field$-vector space and $\set{Y_\alpha}{\alpha \in A}$ a collection of normed vector spaces over $\field$. For each $\alpha \in A$ consider a linear map $T_\alpha \in \calL(X,Y_\alpha)$, and let $\calF = \set{T_\alpha}{\alpha \in A}$. Then $\calF$ of course induces an initial topology on $X$. On the other hand, for each $\alpha \in A$ the map $T_\alpha$ defines a seminorm $p_\alpha$ on $X$ by $p_\alpha(x) = \norm{T_\alpha x}$. We claim that the initial topology on $X$ induced by $\calF$ is the same as the seminorm topology induced by the family $\calP = \set{p_\alpha}{\alpha \in A}$ of seminorms as in \cref{thm:seminorm-topology}.

To see this notice that, for $\alpha \in A$, $x_0 \in X$ and $\epsilon > 0$,
%
\begin{align*}
    \ball[p_\alpha]{x_0}{\epsilon}
        &= \set{ x \in X }{ p_\alpha(x-x_0) < \epsilon } \\
        &= \set[\big]{ x \in X }{ \norm{T_\alpha x-T_\alpha x_0} < \epsilon } \\
        &= T_\alpha\preim \bigl( \ball{T_\alpha x_0}{\epsilon} \bigr).
\end{align*}
%
The initial topology on $X$ induced by $\calF$ is generated by the sets on the right-hand side.\footnote{This is clear if each $T_\alpha$ is surjective. But $T_\alpha \colon X \to Y_\alpha$ is continuous iff the corresponding map with codomain $T_\alpha(X)$ is continuous, so it suffices to consider balls in $Y_\alpha$ with centres in $T(X)$.} On the other hand, the seminorm topology induced by $\calP$ is generated by the sets on the left-hand side. Hence the two topologies agree. In particular, the resulting topology makes $X$ into a locally convex topological vector space.

An important application of the above is when $M$ is a subspace of $\calL(X,Y)$ and $\calF$ is the set of evaluation maps $\ev_x \colon M \to Y$ given by $\ev_x(T) = Tx$ for $x \in X$. It is easy to show that the evaluation maps are in fact linear, and that $\calF$ is even a subspace of $\calL(X,Y)$. We will call the $\calF$-topology on $M$ the \emph{evaluation topology} on $M$.\footnote{This is not standard terminology, but it seems useful to have a name for it.} Since the evaluation maps obviously separate points in $M$, the evaluation topology is Hausdorff (hence $T_3$ by \cref{thm:topological_group_regular}). Notice also that the product topology on $Y^X$ is precisely induced by the evaluation maps, so $M \subseteq Y^X$ in fact carries the subspace topology and is thus a topology of pointwise convergence.


\section{Weak topologies}

Let $X$ be a $\field$-vector space, and let $\calF$ be a collection of linear functionals $X \to \field$. The initial topology on $X$ induced by $\calF$ is called the \emph{$\calF$-topology} on $X$. The above discussion shows that the $\calF$-topology makes $X$ into a locally convex topological vector space. Being an initial topology, the $\calF$-topology is Hausdorff if and only if $\calF$ separates points in $X$.

If $X$ is a topological vector space, then the $\calF$-topology is known as a \emph{weak topology}. The special case where $\calF = X^*$ is simply known as \emph{the} weak topology on $X$, and when equipped with this topology $X$ is sometimes denoted $X_w$. In the case where $X$ is locally convex, \emph{the} weak topology, being an initial topology, is Hausdorff if and only if $X^*$ separates points in $X$, but by \cref{thm:X*-separating-points-Hausdorff} this is the case if and only if the original topology on $X$ is Hausdorff.

In the case where $\calF$ is a vector space of linear functionals we can say even more. First of all, if $X$ is already a topological vector space, then $X$ and $X_w$ clearly have the same continuous linear functionals. That is, the $X^*$-topology on $X$ yields no more continuous functionals on $X$ than those already in $X^*$. This turns out to be a general phenomenon as long as $\calF$ is in fact a vector space. First a lemma:

\begin{lemma}
    \label{thm:linear-functional-span-lemma}
    Let $X$ be a $\field$-vector space, and let $\phi, \phi_1, \ldots, \phi_n$ be linear functionals on $X$. Then the following are equivalent:
    %
    \begin{enumlem}
        \item \label{enum:linear-functional-span-lemma-span} $\phi \in \Span(\phi_1, \ldots, \phi_n)$.
        
        \item \label{enum:linear-functional-span-lemma-bounded} There exists an $\alpha \in \reals$ such that
        %
        \begin{equation*}
            \abs{\phi(x)}
                \leq \alpha \max_{1 \leq i \leq n} \abs{\phi_i(x)}
        \end{equation*}
        %
        for all $x \in X$.

        \item \label{enum:linear-functional-span-lemma-kernel} $\bigintersect_{i=1}^n \ker\phi_i \subseteq \ker\phi$.
    \end{enumlem}
\end{lemma}

\begin{proof}
    It is clear that \subcref{enum:linear-functional-span-lemma-span} implies \subcref{enum:linear-functional-span-lemma-bounded}, and that \subcref{enum:linear-functional-span-lemma-bounded} implies \subcref{enum:linear-functional-span-lemma-kernel}. Assume that \subcref{enum:linear-functional-span-lemma-kernel} holds, and define $\pi \colon X \to \field^n$ by $\pi = \langle \phi_1, \ldots, \phi_n \rangle$, i.e. $\pi(x) = (\phi_1(x), \ldots, \phi_n(x))$ for $x \in X$. Then by assumption $\ker\pi \subseteq \ker\phi$, so there exists a linear functional $f \colon \phi(X) \to \field$ such that $f \circ \pi = \phi$. Extending $f$ to a linear functional $F$ on $\field^n$, there exist $\alpha_1, \ldots, \alpha_n \in \field$ such that
    %
    \begin{equation*}
        F(u_1, \ldots, u_n)
            = \sum_{i=1}^n \alpha_i u_i
    \end{equation*}
    %
    for all $(u_1, \ldots, u_n) \in \field^n$. But then
    %
    \begin{equation*}
        \phi(x)
            = F \circ \pi(x)
            = F(\phi_1(x), \ldots, \phi_n(x))
            = \sum_{i=1}^n \alpha_i \phi_i(x),
    \end{equation*}
    %
    which proves \subcref{enum:linear-functional-span-lemma-span}
\end{proof}


\begin{theorem}
    \label{thm:linear-functionals-create-LCS}
    Let $X$ be a $\field$-vector space, and let $M$ be a vector space of linear functionals on $X$. Then the $M$-topology makes $X$ into a locally convex topological vector space with $X^* = M$.
\end{theorem}

\begin{proof}
    The above discussion shows that $X$ is a locally convex topological vector space, so it suffices to establish the final claim. Clearly $M \subseteq X^*$, so let $\phi \in X^*$ and put $U = \phi\preim(\ball{0}{1})$. This is a neighbourhood of $0$, so it contains a set on the form
    %
    \begin{equation*}
        \ball[\phi_1]{0}{\epsilon}
            \intersect \cdots
            \intersect \ball[\phi_n]{0}{\epsilon}
    \end{equation*}
    %
    for some $\epsilon > 0$ and appropriate $\phi_1, \ldots, \phi_n \in M$. But then property \subcref{enum:linear-functional-span-lemma-bounded} of \cref{thm:linear-functional-span-lemma} holds, implying that $\phi \in \Span(\phi_1, \ldots, \phi_n) \subseteq M$.
\end{proof}


\begin{proposition}
    Let $X$ be a locally convex topological vector space, and let $C \subseteq X$ be convex. Then $C$ is closed if and only if it is closed in the weak topology.
\end{proposition}

\begin{proof}
    Since the weak topology is coarser than the original topology on $X$, a weakly closed set is also closed. So assume that $C$ is closed in the original topology, and let $a \in X \setminus C$. Since $\{a\}$ is convex and compact, \cref{enum:Hahn-Banach-separation-compact-closed} then furnishes a $\phi \in X^*$ such that
    %
    \begin{equation*}
        \Re \phi(a)
            < \inf_{c \in C} \Re \phi(c).
    \end{equation*}
    %
    The map $x \mapsto \Re \phi(x)$ is weakly continuous at $a$, so $a$ has a weakly open neighbourhood disjoint from $C$. Hence $X \setminus C$ is weakly open, so $C$ is weakly closed.
\end{proof}


\section[The weak*-topology][The weak$^*$-topology]{The weak$^*$-topology}

Let $X$ be a topological vector space. The evaluation topology on $X^*$ is called the \emph{weak$^*$-topology} on $X^*$. Since the evaluation maps constitute a vector space, \cref{thm:linear-functionals-create-LCS} implies that $X^*$ with the weak$^*$-topology is a locally convex Hausdorff topological vector space, and that every weak$^*$-continuous linear functional on $X^*$ is on the form $\ev_x$ for some $x \in X$.

The most important property of the weak$^*$-topology is the following:

\begin{theorem}[The Banach--Alaoglu Theorem]
    If $X$ is a normed vector space, then the closed unit ball $B^* = \set{\phi \in X^*}{\norm{\phi} \leq 1}$ in $X^*$ is compact in the weak$^*$-topology.
\end{theorem}

\begin{proof}
    For $x \in X$ let $D_x = \set{z \in \field}{\abs{z} \leq \norm{x}}$. Then $D = \bigprod_{x \in X} D_x$ consists of those $\phi \in \field^X$ with $\abs{\phi(x)} \leq \norm{x}$ for all $x \in X$, and $B^*$ is the subset of $D$ of linear maps. Since $D$ is compact by Tychonoff's theorem, it suffices to show that $B^*$ is closed in $D$. If $(\phi_i)_{i \in I}$ is a net in $B^*$ that converges to some $\phi \in D$, then for $x,y \in X$ and $\alpha \in \field$ we have
    %
    \begin{equation*}
        \phi(\alpha x + y)
            = \lim_{i \in I} \phi_i(\alpha x + y)
            = \lim_{i \in I} \bigl( \alpha \phi_i(x) + \phi_i(y) \bigr)
            = \alpha \phi(x) + \phi(y),
    \end{equation*}
    %
    since $D$ has the topology of pointwise convergence. Hence $\phi \in B^*$ as desired.
\end{proof}


\section{The strong operator topology}

Now let $X$ and $Y$ be normed vector spaces, and let $\calB(X,Y)$ be the space of bounded linear maps $X \to Y$. The evaluation topology on $\calB(X,Y)$ is called the \emph{strong operator topology} (or simply \enquote{SOT}). More concretely, the topology is induced by evaluation maps
%
\begin{equation*}
    \ev_x \colon \calB(X,Y) \to Y
\end{equation*}
%
for $x \in X$. Hence it is generated by seminorms $T \mapsto \norm{Tx}$, so a net $(T_i)_{i \in I}$ in $\calB(X,Y)$ converges to $T$ iff $\norm{T_i x - Tx} \to 0$ for all $x \in X$ (of course this only verifies that evaluation topologies are topologies of pointwise convergence).
    
Notice that the SOT is coarser than the norm topology, since if $T_i \to T$ in the norm topology, then
%
\begin{equation*}
    \norm{T_i x - Tx}
        \leq \norm{T_i - T} \, \norm{x}
        \to 0,
\end{equation*}
%
so $T_i \to T$ in the SOT.


\section{The weak operator topology}

Again let $X$ and $Y$ be normed vector spaces. The \emph{weak operator topology} (or simply \enquote{WOT}) on $\calB(X,Y)$ is a variant of the strong operator topology, where $Y$ is given the weak topology. More precisely, the weak operator topology is induced by evaluation maps
%
\begin{equation*}
    \ev_x \colon \calB(X,Y) \to Y_w
\end{equation*}
%
for $x \in X$. Thus the WOT is not strictly speaking an evaluation topology as defined above. Since $Y_w$ itself has the initial topology induced by $Y^*$, the WOT is the initial topology induced by the linear maps $\phi \circ \ev_x$ for $x \in X$ and $\phi \in Y^*$, or equivalently by the seminorms $T \mapsto \phi(Tx)$. Hence a net $(T_i)_{i \in I}$ in $\calB(X,Y)$ converges to $T$ iff $\phi(T_i x) \to \phi(Tx)$ for all $x \in X$ and $\phi \in Y^*$.

Clearly the WOT is coarser than the SOT, since if $\ev_x$ is continuous then so is $\phi \circ \ev_x$.

We claim that the WOT is also Hausdorff. This is not immediate from the above since the generating functions are not evaluation maps. But notice that $Y^*$ separates points in $Y$ by \cref{thm:X*-separating-points-Hausdorff}. For distinct $T,S \in \calL(X,Y)$ there is a $x \in X$ with $Tx \neq Sx$, and then a $\phi \in Y^*$ with $\phi(Tx) \neq \phi(Sx)$. Hence the functions $\phi \circ \ev_x$ separate points in $\calB(X,Y)$.

If $\calH$ is a Hilbert space, then the weak operator topology on $\calB(X,\calH)$ is also induced by maps $T \mapsto \inner{Tx}{y}$ by the Riesz--Fréchet theorem. In this case a net $(T_i)$ converges to $T$ iff $\inner{T_i x}{y} \to \inner{Tx}{y}$ for all $x \in X$ and $y \in \calH$.


\nocite{*}

\printbibliography


\end{document}
