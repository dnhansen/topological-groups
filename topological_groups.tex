% Document setup
\documentclass[article, a4paper, 11pt, oneside]{memoir}
\usepackage[utf8]{inputenc}
\usepackage[T1]{fontenc}
\usepackage[UKenglish]{babel}

% Document info
\newcommand\doctitle{Topological Groups}
\newcommand\docauthor{Danny Nygård Hansen}

% Formatting and layout
\usepackage[autostyle]{csquotes}
\usepackage[final]{microtype}
\usepackage{xcolor}
\frenchspacing
\usepackage{latex-sty/articlepagestyle}
\usepackage{latex-sty/articlesectionstyle}

% Fonts
\usepackage[largesmallcaps]{kpfonts}
\DeclareSymbolFontAlphabet{\mathrm}{operators} % https://tex.stackexchange.com/questions/40874/kpfonts-siunitx-and-math-alphabets
\linespread{1.06}
\let\mathfrak\undefined
\usepackage{eufrak}
\usepackage{inconsolata}
\usepackage{amssymb}

% Hyperlinks
\usepackage{hyperref}
\definecolor{linkcolor}{HTML}{4f4fa3}
\hypersetup{%
	pdftitle=\doctitle,
	pdfauthor=\docauthor,
	colorlinks,
	linkcolor=linkcolor,
	citecolor=linkcolor,
	urlcolor=linkcolor,
	bookmarksnumbered=true
}

% Equation numbering
\numberwithin{equation}{chapter}

% Footnotes
\footmarkstyle{\textsuperscript{#1}\hspace{0.25em}}

% Mathematics
\usepackage{latex-sty/basicmathcommands}
\usepackage{latex-sty/framedtheorems}
\usepackage{latex-sty/topologycommands}
\usepackage{tikz-cd}
\usetikzlibrary{babel}

% Lists
\usepackage{enumitem}
\setenumerate[0]{label=\normalfont(\arabic*)}

% Bibliography
\usepackage[backend=biber, style=authoryear, maxcitenames=2, useprefix]{biblatex}
\addbibresource{references.bib}

% Title
\title{\doctitle}
\author{\docauthor}

\newcommand{\calT}{\mathcal{T}}


% Section style -- add to section style .sty?
\setsubsecheadstyle{\normalfont\itshape}


% Preimage -- to be added to mathcommands .sty
\newcommand{\preim}{^{-1}}


\begin{document}

\maketitle

\chapter{Introduction}

The purpose of these notes is to clarify some of the basic properties of topological groups. We in particular hope to shed light on the reasonability of the common assumption that topological groups be Hausdorff.

Most of the results below can be found in various forms in a variety of books or articles (see the references), but I have not come across a resource that develops this theory in precisely this way. One example is the $T_0$-identification of a topological group, as we will see below.


\chapter{Definitions and basic properties}

\begin{definition}[Topological groups]
    \label{def:topological_group}
    A \emph{topological group} is a triple $(G, \mu, \calT)$ such that $(G, \mu)$ is a group, $(G, \calT)$ is a topological space, and both the multiplication $\mu \colon G \prod G \to G$ and the inversion map $\iota \colon G \to G$ given by $g \mapsto g\inv$ are continuous.
\end{definition}
%
In the sequel we will simply write $G$, and it will be clear from context whether $G$ is to be thought of as a set, group, topological space or topological group. It is custom to assume that a topological group be $T_1$ (or Hausdorff, which is actually implied by it being $T_1$, as will follow from \cref{thm:topological_group_regular}), but we omit this assumption since our focus is precisely on the basic topological properties of topological groups.

\newcommand{\leftmult}{\lambda}
\newcommand{\rightmult}{\rho}

Let $G$ be a topological group and let $g \in G$. We define maps $\leftmult_g, \rightmult_g \colon G \to G$ by $\leftmult_g(h) = gh$ and $\rightmult_g(h) = hg$; that is, $\leftmult_g$ and $\rightmult_g$ are given by left- and right-multiplication by $g$, respectively. Both of these are homeomorphisms, with $\leftmult_g\inv = \leftmult_{g\inv}$ and $\rightmult_g\inv = \rightmult_{g\inv}$. Hence if $a, b \in G$, then there is a homeomorphism on $G$ that takes $a$ to $b$, namely $\leftmult_{ba\inv}$. Thus a topological group is \emph{homogeneous} as a topological space.

We begin by collecting some basic properties of topological groups:

\begin{proposition}[Properties of topological groups]
    Let $G$ be a topological group, and let $A, B \subseteq G$.
    %
    \begin{enumprop}
        \item \label{enum:closure_of_product} $\closure{A} \closure{B} \subseteq \closure{AB}$.

        \item \label{enum:product_open_set} If $U \subseteq G$ is open, then $AU$ and $UA$ are open.
        
        \item \label{enum:elements-of-closure-commute} Assume that $G$ is Hausdorff. If $ab = ba$ for all $a \in A$, $b \in B$, then $ab = ba$ for all $a \in \closure{A}$, $b \in \closure{B}$.
    \end{enumprop}
\end{proposition}
%
As we will see in \cref{thm:topological_group_regular}, the Hausdorff assumption in \subcref{enum:elements-of-closure-commute} can be weakened to just $T_0$. However, this assumption cannot be dropped: Any non-trivial, nonabelian group is a topological group in the trivial topology, but then the closure of the trivial subgroup is the whole group and hence not abelian.

\begin{proof}
    We give two proofs of \subcref{enum:closure_of_product}. Since the multiplication $\mu \colon G \prod G \to G$ is continuous we get
    %
    \begin{equation*}
        \mu(\closure{A}, \closure{B})
            = \mu(\closure{A} \prod \closure{B})
            = \mu(\closure{A \prod B})
            \subseteq \closure{\mu(A \times B)}
            = \closure{\mu(A, B)}
            = \closure{AB}.
    \end{equation*}
    %
    Alternatively, consider $a \in \closure{A}$ and $b \in \closure{B}$. If $U$ is a neighbourhood of $ab$, then by continuity of multiplication there are neighbourhoods $V_1$ and $V_2$ of $a$ and $b$ respectively such that $V_1 V_2 \subseteq U$. Picking $x \in A \intersect V_1$ and $y \in B \intersect V_2$ we find that $xy \in AB \intersect U$, so $ab \in \closure{AB}$.

    To prove \subcref{enum:product_open_set}, notice that e.g.
    %
    \begin{equation*}
        AU = \bigunion_{g \in A} gU,
    \end{equation*}
    %
    so $AU$ is a union of open sets.

    Let $c_a \colon G \to G$ be conjugation by $a$, i.e. $c_a(g) = aga\inv$. Then \subcref{enum:elements-of-closure-commute} says that if $c_a$ is the identity on $B$, then it is the identity on $\closure{B}$. But since $G$ is Hausdorff, this follows.
\end{proof}

We next remark that the assumption that $G$ be $T_1$ can be weakened:

\begin{proposition}[$T_0$ implies $T_1$]
    \label{thm:T0_implies_T1}
    Let $G$ be a topological group. If $G$ is $T_0$, then it is in fact $T_1$.
\end{proposition}

\newcommand{\calN}{\mathcal{N}}
\DeclarePairedDelimiter{\nhoodfilteraux}{(}{)}
% \newcommand{\nhoodfilter}[1]{\calN\nhoodfilteraux{#1}}
\newcommand{\nhoodfilter}[1]{\calN_{#1}}

\begin{proof}
    Assume that $G$ is $T_0$. By homogeneity it suffices to show that the singleton $\{e\}$ containing the identity $e \in G$ is closed. We show that $G \setminus \{e\}$ is a neighbourhood\footnotemark{} of all its points. Let $g \neq e$. Since $G$ is $T_0$, either $G \setminus \{e\}$ is a neighbourhood of $g$ or $G \setminus \{g\}$ is a neighbourhood of $e$. In the first case we are done, so assume the latter. The homeomorphism $\leftmult_{g\inv}$ maps $G \setminus \{g\}$ to $G \setminus \{e\}$, so the latter set is a neighbourhood of $g\inv$. But the inversion map $\iota$ is a homeomorphism that maps $G \setminus \{e\}$ to itself, so this is also a neighbourhood of $g$. This proves the claim.
\end{proof}
\footnotetext{If $X$ is a topological space and $A \subseteq X$, then we say that a set $N \subseteq X$ is a \emph{neighbourhood} of $A$ if there is an open set $U$ in $X$ such that $A \subseteq U \subseteq N$. The family of neighbourhoods of a set $A$ is called the \emph{neighbourhood filter of $A$} and is denoted $\nhoodfilter{A}$. If $A = \{x\}$ is a singleton we also write $\nhoodfilter{x}$ and call $N$ a neighbourhood of $x$.}

Recall that a topological space $X$ is called \emph{regular} if it is possible to separate any point from any closed set by disjoint open sets.\footnote{In our terminology, a regular $T_1$-space would be called a \emph{$T_3$-space}.} Our next order of business is to show that any topological group is regular. Before proving this we need some terminology and a lemma:

If $G$ is a topological group and $A \subseteq G$, then we write
%
\begin{equation*}
    A\inv = \iota(A) = \set{a\inv}{a \in A}.
\end{equation*}
%
A subset $A$ is called \emph{symmetric} if $A = A\inv$. Notice that since $\iota$ is a homeomorphism, $A$ is open (closed) if and only if $A\inv$ is open (closed).

\begin{lemma}
    \label{thm:symmetric_nhood_of_e}
    Let $G$ be a topological group. If $U$ is a neighbourhood of the identity $e$, then there is a symmetric neighbourhood $V$ of $e$ such that $VV \subseteq U$.
\end{lemma}

\begin{proof}
    Since multiplication is continuous and $ee = e$, there are neighbourhoods $V_1$ and $V_2$ of $e$ such that $V_1 V_2 \subseteq U$. If we let
    %
    \begin{equation*}
        V = V_1 \intersect V_2 \intersect V_1\inv \intersect V_2\inv,
    \end{equation*}
    %
    then $V$ has the desired properties.
\end{proof}


\begin{proposition}[Regularity of topological groups]
    \label{thm:topological_group_regular}
    If $G$ is a topological group, $g \in G$, $A \subseteq G$ is closed and $g \not\in A$, then there exists a symmetric neighbourhood $V$ of $e$ such that $Vg$ and $VA$ are disjoint. In particular every topological group is regular, and every $T_0$ topological group is $T_3$.
\end{proposition}

\begin{proof}
    Since $g \not\in A$ we also have $e \not\in Ag\inv$. But $Ag\inv$ is closed, so by \cref{thm:symmetric_nhood_of_e} there is a symmetric neighbourhood $V$ of $e$ such that $VV \intersect Ag\inv = \emptyset$. It follows that $Vg \intersect VA = \emptyset$ as desired. This shows that $G$ is regular, and the final claim follows by \cref{thm:T0_implies_T1}.
\end{proof}


\chapter{Subgroups}

[Algebraic subgroup is topological group in subspace topology.]

\begin{proposition}[Properties of subgroups]
    Let $G$ be a topological group.
    %
    \begin{enumprop}
        \item \label{enum:closure_of_subgroup} If $H$ is a subgroup of $G$, then so is $\closure{H}$. If $H$ is normal, then so is $\closure{H}$. If $G$ is $T_0$ and $H$ is abelian, then so is $\closure{H}$.
        
        \item \label{enum:open_subgroup_is_closed} Every open subgroup of $G$ is closed.
    \end{enumprop}
\end{proposition}

\begin{proof}
    Let $H$ be a subgroup of $G$. By \cref{enum:closure_of_product} we get
    %
    \begin{equation*}
        \closure{H} \closure{H} \subseteq \closure{HH} = \closure{H},
    \end{equation*}
    %
    so $\closure{H}$ is closed under multiplication. It is also closed under taking inverses, since the inverse map $\iota$ is a homeomorphism. Hence it is a subgroup. Alternatively, this claim is an easy consequence of basic properties of nets.

    Clearly $\closure{H}$ is the smallest closed subgroup that contains $H$. Hence if $H$ is normal and $\closure{H}$ is not, then intersecting $\closure{H}$ with one of its conjugates yields a strictly smaller closed subgroup containing $H$.

    If $G$ is $T_0$ and $H$ is abelian, then it follows from \cref{enum:elements-of-closure-commute} with $A = B = H$ that $\closure{H}$ is abelian. This proves \subcref{enum:closure_of_subgroup}.

    Finally let $H$ be an open subgroup. Since $G$ is the disjoint union of the left cosets of $H$, we have
    %
    \begin{equation*}
        G \setminus H
            = \bigunion_{g \in G \setminus H} gH.
    \end{equation*}
    %
    Since the cosets $gH$ are open, $G \setminus H$ is also open. This proves \subcref{enum:open_subgroup_is_closed}.
\end{proof}



\chapter{Coset spaces and quotient groups}

\section{General properties of coset spaces}

If $H$ is a subgroup of a topological group $G$, we denote by $G/H$ the set of left cosets of $H$. Let $q \colon G \to G/H$ be the quotient map and equip $G/H$ with the quotient topology. We call $G/H$ a \emph{(left) coset space} of $G$ by $H$.

Notice that $q$ is in fact open: If $U \subseteq G$ is open, then $q\preim(q(U)) = UH$ is also open, so $q(U)$ is open since $G/H$ has the quotient topology coinduced by $q$.


\begin{proposition}[Properties of coset spaces]
    Let $G$ be a topological group and $H$ a subgroup.
    %
    \begin{enumprop}
        \item \label{enum:coset_space_regular} The coset space $G/H$ is regular.

        \item The topology on $G/H$ is homogeneous.
        
        \item \label{enum:coset_space_T1} $G/H$ is $T_1$ (and hence $T_3$) if and only if $H$ is closed.

        \item $G/H$ is discrete if and only if $H$ is open.
        
        \item \label{enum:coset_space_locally_compact} If $G$ is locally compact\footnotemark, then so is $G/H$.
    \end{enumprop}
\end{proposition}\footnotetext{We say that a topological space is locally compact if every point has a compact neighbourhood. There are many non-equivalent definitions of local compactness and this is the least restrictive one. If $H$ is closed then $G/H$ is Hausdorff, so all the usual definitions of local compactness are equivalent for $G/H$.}
%
Notice that \subcref{enum:coset_space_T1} does not assume any separation properties of $G$.

(The claim \subcref{enum:coset_space_regular} is the content of supplementary exercise 7(d) of \textcite[Chapter~2]{munkres}, except that in the exercise $H$ is assumed to be closed. I do not see where this assumption is used.)

\begin{proof}
    We first show that $G/H$ is regular. Let $q(x) \in G/H$, and let $B \subseteq G/H$ be a closed set. Then $A = q\preim(B)$ is closed, so by \cref{thm:topological_group_regular} there is a symmetric neighbourhood $V$ of $e$ in $G$ such that $Vx \intersect VA = \emptyset$. Since $A$ is a union of left cosets of $H$ we have $A = AH$, and a quick calculations shows that $VxH \intersect VA = \emptyset$. It follows by a similar calculation that $q(Vx)$ and $q(VA)$ are disjoint, and since $q$ is open these are neighbourhoods of $q(x)$ and $q(A) = B$ respectively.

    Next we show that $G/H$ is homogeneous. For $g \in G$ define a map $\theta_g \colon G/H \to G/H$ by $\theta(q(x)) = q(gx)$. This is well-defined, since if $xH = yH$, then $gxH = gyH$. Furthermore, the diagram
    %
    \begin{equation*}
        \begin{tikzcd}
            G
                \ar[r, "\leftmult_g"]
                \ar[d, "q", swap]
            & G
                \ar[d, "q"] \\
            G/H
                \ar[r, "\theta_g", swap]
            & G/H
        \end{tikzcd}
    \end{equation*}
    %
    commutes, so $\theta_g \circ q$ is continuous. But by the characteristic property of the quotient topology on $G/H$, $\theta_g$ is also continuous. Since $\theta_g\inv = \theta_{g\inv}$ it is in fact a homeomorphism, and $\theta_{yx\inv}$ takes $q(x)$ to $q(y)$, which shows homogeneity.

    Next we show that $G/H$ is $T_1$ if and only if $H$ is closed. Note that fibres of $q$ are cosets of $H$, and a coset $gH = \leftmult_g(H)$ is closed if and only if $H$ is. But since $G/H$ carries the quotient topology, $gH$ is closed in $G$ if and only if $\{gH\}$ is closed in $G/H$. Similarly, $H$ (and hence $gH$) is \emph{open} in $G$ if and only if $\{gH\}$ is open in $G/H$, i.e. if and only if $G/H$ is discrete.

    % Now assuming that $H$ is closed, we show that $G/H$ is Hausdorff. Let $xH$ and $yH$ be distinct (and hence disjoint) cosets. Then $xHy\inv$ is a closed set not containing $e$, so \cref{thm:symmetric_nhood_of_e} implies the existence of a symmetric neighbourhood $U$ of $e$ such that $UU \intersect xHy\inv = \emptyset$. It follows that
    % %
    % \begin{equation*}
    %     e
    %         \not\in U xHy\inv U
    %         = Ux H (Uy)\inv
    %         = (UxH)(UyH)\inv,
    % \end{equation*}
    % %
    % where we use that $U = U\inv$ and $H = HH$. That $e$ does not lie in the left-most set is easily proven e.g. by contraposition. It follows that $UxH$ and $UyH$ are disjoint, and since $q$ is open this implies that $q(Ux)$ and $q(Uy)$ are disjoint neighbourhoods of $xH$ and $yH$ in $G/H$. This proves \subcref{enum:coset_space_hausdorff}.

    To prove \subcref{enum:coset_space_locally_compact}, notice that if $K$ is a compact neighbourhood of $e$ in $G$, then $q(Kx)$ is a compact neighbourhood of $xH$ in $G/H$.
\end{proof}

% This proposition also furnishes a different proof that $T_1$ implies Hausdorff for topological groups: Simply let $H = \{e\}$, in which case $G \cong G/H$.


\section{Topological quotient groups}

If a subgroup $N$ of a topological group $G$ is normal, we expect that the usual group structure on the (algebraic) quotient group $G/N$ is compatible with the quotient topology. This is indeed the case:

\begin{theorem}[Topological quotient groups]
    \label{thm:topological-quotient-group}
    If $N$ is a normal subgroup of a topological group $G$, then $G/N$ is a topological group.
\end{theorem}

\begin{proof}
    If $x,y \in G$ and $U$ is a neighbourhood of $(xN)(yN) = xyN$ in $G/N$, then continuity of multiplication in $G$ at $(x,y)$ implies the existence of neighbourhoods $V$ and $W$ of $x$ and $y$ respectively, such that $VW \subseteq q\preim(U)$. Since $q$ is surjective it follows that $q(V) q(W) \subseteq U$, and because $q$ is also open $q(V)$ and $q(W)$ are neighbourhoods of $xN$ and $yN$. Hence multiplication is continuous.

    Since the inversion map $\iota$ on $G/N$ is bijective, it suffices to show that it is open. Let $U \subseteq G/N$ be open and notice that, since $q$ is surjective,
    %
    \begin{equation*}
        \iota(U)
            = \iota \bigl( q(q\preim(U)) \bigr)
            = q \bigl( \iota(q\preim(U)) \bigr).
    \end{equation*}
    %
    Because $q\preim(U)$, and hence $\iota(q\preim(U))$, is open in $G$, it follows that $\iota(U)$ is open since $q$ is open.
\end{proof}

We now explore how a topological group $G$ that is \emph{not} $T_0$ can be made so by quotienting out by a particular subgroup. To do this justice we first recall the \emph{$T_0$-identification} of a topological space $X$: The ordering $x \leq y$ defined by $x \in \closure{\{y\}}$ is called the \emph{specialisation preorder}, and it is easy to show that $x \leq y$ is equivalent to $\nhoodfilter{x} \subseteq \nhoodfilter{y}$. This order gives rise to an equivalence relation $\equiv$, and we say that two points $x,y \in X$ are \emph{topologically indistinguishable} if $x \equiv y$.

It is clear that $X$ is $T_0$ if and only if the relation $\equiv$ is trivial, and it is not difficult to show that the quotient space $X/{\equiv}$, called the \emph{$T_0$-identification} or the \emph{Kolmogorov quotient} of $X$, is indeed $T_0$. In fact, $\equiv$ is the most conservative equivalence relation $\sim$ such that $X/{\sim}$ is $T_0$, though we shall not need this fact. [reference to separation axiom notes]

When we apply this construction to the theory of topological groups, we see that the $T_0$-identification can be understood in terms of quotient groups:

\begin{proposition}[$T_0$-identification of groups]
    \label{thm:quotient-group-T0-identification}
    Let $G$ be a topological group. The subgroup $\closure{\{e\}}$ of $G$ is normal. Furthermore, the $\equiv$-equivalence classes are precisely the left cosets of $\closure{\{e\}}$. It follows that the quotient group $G / \closure{\{e\}}$ is just the $T_0$-identification $G/{\equiv}$ of $G$.
\end{proposition}

\begin{proof}
    The subgroup $\closure{\{e\}}$ is the smallest closed subgroup of $G$, hence it is normal since otherwise intersecting it with one of its conjugates yields a strictly smaller closed subgroup.

    It then suffices to show that, for $x,y \in G$, $x \equiv y$ if and only if $x \closure{\{e\}} = y \closure{\{e\}}$. But this is clear since e.g. $x \closure{\{e\}} = \closure{\{x\}}$ by continuity of multiplication on $G$.
\end{proof}

It is results like the above that lead to the common assumption that topological groups are $T_1$, since if it is not then we just quotient out by $\closure{\{e\}}$. One might justify this by arguing as follows: If a topology on a set $X$ is to respect the set structure, then there must be a difference between two distinct points that is captured by the topology. But this difference must lie in which neighbourhoods each points has; the topology simply carries no further information than this. So if $x \neq y$, then it must be the case that $x \not\equiv y$, i.e. that $X$ is $T_0$.

On the other hand, a \emph{group} structure on a set $G$ certainly respects the set structure, in that different elements may give rise to different actions: just let $G$ act on itself by multiplication.

Thus if the topology and group structure on a topological group are to be truly compatible, then the topology must respect the underlying set structure, i.e. be $T_0$. In the sequel we shall, however, resist assuming that topological groups are $T_0$ as far as possible.

There is more to say about the $T_0$-identification that is not pertinent to this discussion. We refer to \textcite[Exercise~13C]{willard} and \textcite{pirttimäki2019}. [And my own notes]


\section{Metrics}

Next we consider metrics: A pseudo-metric $\rho$ on a group $G$ is called \emph{(translation) invariant} if $\rho(xz,yz) = \rho(x,y)$ for all $x,y,z \in G$.

Assume that $G$ is pseudo-metrisable by an invariant pseudo-metric $\rho$. Let $H$ be a subgroup of $G$ and define $\rho_H \colon G/H \times G/H \to [0,\infty)$ by
%
\begin{equation*}
    \rho_H(q(x), q(y))
        = \inf_{h \in H} \rho(y\inv x, h).
\end{equation*}
%
This is well-defined, since if $q(x) = q(x')$ and $q(y) = q(y')$, then $x\inv x' = h_1$ and $y\inv y' = h_2$ for appropriate choices of $h_1,h_2 \in H$, and so
%
\begin{equation*}
    \rho(y\inv x, h)
        = \rho((y')\inv x', h_2\inv h h_1).
\end{equation*}
%
We show below that $\rho_H$ is a pseudo-metric, and we call it the \emph{quotient metric} on $G/H$.

We can interpret the quotient metric as follows: We think of the elements of a group as invertible transformations. Since $H$ is the identity in $G/H$, $\inf_{h \in H} \rho(y\inv x, h)$ measures the distance from $y\inv x$ to the closest transformation that \textquote{does nothing}. If this number is small, then $y\inv x$ can be thought of \textquote{doing almost nothing}. That is, first performing the transformation $x$ and then undoing the transformation $y$ we do almost nothing, so $x$ and $y$ must be almost the same transformation.

This kind of reasoning also shows why we are only interested in invariant metrics: Otherwise $y\inv x$ being close to $e$ would not immediately imply that $x$ and $y$ are close, so a non-invariant metric would not respect the interpretation of group elements as transformations.


\begin{proposition}[Quotient metric]
    \label{thm:coset-space-metric}
    The map $\rho_H$ is a pseudo-metric on $G/H$ compatible with the quotient topology. If $H$ is normal such that $G/H$ is a group, then $\rho_H$ is invariant. Furthermore, $\rho_H$ is a metric if and only if $H$ is closed.
\end{proposition}

\newcommand{\calB}{\mathcal{B}}

\begin{proof}
    It is clear that $\rho_H$ is positive and symmetric. We prove the triangle inequality: Let $x,y,z \in G$. For any $h, h' \in H$ we find that
    %
    \begin{align*}
        \rho_H(q(x), q(z))
            &\leq \rho(z\inv x, hh') \\
            &\leq \rho(z\inv x, z\inv yh') + \rho(z\inv yh', hh') \\
            &= \rho(y\inv x, h') + \rho(z\inv y, h).
    \end{align*}
    %
    By taking infima over $h,h' \in H$ on the right-hand side we thus get
    %
    \begin{equation*}
        \rho_H(q(x),q(z))
            \leq \rho_H(q(x),q(y)) + \rho_H(q(y),q(z))
    \end{equation*}
    %
    as desired. It is clear from the definition that $\rho_H$ is invariant if $H$ is normal.

    Next we show that $\rho_H$ is compatible with the quotient topology on $G/H$. Let $\calB$ be a local basis at $x \in G$. We claim that $q(\calB)$ is a local basis at $q(x)$. If $U \subseteq G/H$ is a neighbourhood of $q(x)$, then $q\preim(U)$ is a neighbourhood of $x$. Thus there is a basic neighbourhood $B \in \calB$ such that $B \subseteq q\preim(U)$. Since $q$ is open, it follows that $q(B) \subseteq U$ is a neighbourhood of $q(x)$.

    Now we claim that $\rho(x,y) < r$ if and only if $\rho_H(q(x),q(y)) < r$, for $r > 0$. The latter is true precisely when $\rho(xy\inv, h) < r$ for all $h \in H$, so in particular when $h = e$, i.e. when $\rho(x,y) < r$. Conversely, if $\rho(x,y) < r$ then
    %
    \begin{equation*}
        \rho_H(q(x),q(y))
            \leq \rho(xy\inv, e)
            = \rho(x,y)
            < r.
    \end{equation*}
    %
    Fix $x \in G$ and notice that $\set{B_\rho(x,r)}{r > 0}$ is a local basis at $x$. Since $q(B_\rho(x,r)) = B_{\rho_H}(q(x),r)$ by the argument above, the set $\set{B_{\rho_H}(q(x),r)}{r > 0}$ is a local basis at $q(x)$. Hence $\rho_H$ is compatible with the topology on $G/H$.

    Finally, recall that a pseudo-metric is a metric if and only if the metric topology is $T_0$. By \cref{thm:T0_implies_T1} this is equivalent to it being $T_1$, and \cref{enum:coset_space_T1} implies that this is the case if and only if $H$ is closed.
\end{proof}

[Metric identification]


\chapter{Topological vector spaces}

The purpose of these notes is not to give a complete discussion of either topological groups or vector spaces. In this section we will thus focus on the very basic aspects of topological vector spaces and refer the reader to the literature for further developments.

[Definition of TVS.]

\section{Quotient spaces}

We begin by reviewing the basic construction and properties of quotients of general \emph{algebraic} vector spaces. So let $X$ be a vector space over a field $k$, and let $S$ be a subspace of $X$. Then $X$ is in particular an abelian group under addition, and $S$ is a subgroup. We may then construct the quotient \emph{group} $X/S$ whose elements are the cosets $x + S$ for $x \in X$. Let $\pi \colon X \to X/S$, so that $\pi(x) = x + S$.

The equivalence relation defining the group is also compatible with the scalar multiplication on $X$: For if $x,y \in X$ and $x - y \in S$, then for $r \in k$ we have
%
\begin{equation*}
    rx - ry
        = r(x - y) \in S,
\end{equation*}
%
since $S$ is a subspace. Thus $\pi(rx) = \pi(ry)$, and we may define multiplication by $r \in k$ in $X/S$ by $r \pi(x) = \pi(rx)$. It is routine to check that this makes $X/S$ into a $k$-vector space, called the \emph{quotient space} of $X$ by $S$. It is easy to see that $\pi$ is linear, and that $\ker \pi = S$.

\newcommand{\setF}{\mathbb{F}}

We now specialise to topological vector spaces. Let $X$ be a topological vector space over the field $\setF$. If $S$ is a subspace of $X$, then recall that we give $X/S$ the quotient topology coinduced by the quotient map $\pi$, and that $\pi$ is open.


\begin{theorem}[Topological quotient spaces]
    Let $X$ be a topological vector space, and let $S$ be a subspace. Then $X/S$ is a topological vector space.
\end{theorem}

\begin{proof}
    By \cref{thm:topological-quotient-group} it suffices to show that scalar multiplication is continuous. Let $x \in X$ and $\alpha \in \setF$, and let $U$ be a neighbourhood of $\pi(\alpha x)$ in $X/S$. Continuity of scalar multiplication in $S$ at $(\alpha, x)$ implies the existence of neighbourhoods $V \subseteq \setF$ and $W \subseteq X$ of $\alpha$ and $x$ respectively, such that $VW \subseteq \pi\preim(U)$. Since $\pi$ is surjective and linear, it follows that $V \pi(W) \subseteq U$, and since $\pi$ is also open $\pi(W)$ is a neighbourhood of $\pi(x)$. Hence scalar multiplication in $X/S$ is continuous.
\end{proof}
%
Notice that the second part of the proof of \cref{thm:topological-quotient-group} is redundant for topological vector spaces. One may thus omit this part if one is only interested in vector spaces and not in groups in their own right.

We have already considered quotient metrics on arbitrary topological groups. Here is an important example of a quotient metric:

\newcommand{\calL}{\mathcal{L}}
\newcommand{\calE}{\mathcal{E}}

\begin{example}[$\calL^p(\mu)$- and $L^p(\mu)$-spaces]
    Let $(X, \calE, \mu)$ be a measure space, and consider the Lebesgue space $\calL^p(\mu)$ of $p$-integrable functions for some $p \in (0,\infty)$. This is given the metric topology induced by the pseudo-metric
    %
    \begin{equation*}
        \rho(f,g)
            = \int_X \abs{f-g}^p \dif\mu.
    \end{equation*}
    %
    Since $\calL^p(\mu)$ is not $T_0$, we wish to consider its $T_0$-identification. By \cref{thm:quotient-group-T0-identification} we can find this by computing the closure $N$ of the zero function. This is easily seen to be precisely the space of functions that are zero $\mu$-a.e. We denote the quotient space by $L^p(\mu) = \calL^p(\mu)/N$. We easily see that the quotient metric is simply given by $\rho_N([f],[g]) = \rho(f,g)$, which is the same metric as in the metric identification.
\end{example}

Next we consider norms: Let $\norm{\,\cdot\,}$ be a semi-norm on $X$. We define a map $\norm{\,\cdot\,}_S \colon X/S \to [0, \infty)$ by
%
\begin{equation*}
    \norm{\pi(x)}_S
        = \inf_{s \in S} \norm{x - s}.
\end{equation*}

\begin{proposition}[Quotient norm]
    The map $\norm{\,\cdot\,}_S$ is a semi-norm on $X/S$. If $\rho$ is the pseudo-metric induced by the semi-norm on $X$ and $\rho_S$ is the corresponding quotient metric on $X/S$, then
    %
    \begin{equation*}
        \norm{\pi(x)}_S = \rho_S(\pi(x), 0).
    \end{equation*}
    %
    In particular, $\rho_S$ is the pseudo-metric induced by $\norm{\,\cdot\,}_S$, and $\norm{\,\cdot\,}_S$ is compatible with the quotient topology. Furthermore, it is a norm if and only if $S$ is closed.
\end{proposition}

\begin{proof}
    It is clear that $\norm{\,\cdot\,}_S$ is positive and absolutely homogeneous. We next prove the triangle inequality:
    
    Let $\rho$ be the metric induced by the semi-norm on $X$, i.e. $\rho(y,z) = \norm{y - z}$ for $y,z \in X$. Notice that since $\rho$ is invariant, for $x \in X$,
    %
    \begin{align*}
        \norm{\pi(x)}_S
            &= \inf_{s \in S} \norm{x - s}
             = \inf_{s \in S} \rho(x - s, 0) \\
            &= \inf_{s \in S} \rho(x - 0, s)
             = \rho_S(\pi(x), 0).
    \end{align*}
    %
    If also $y \in X$, then it follows that
    %
    \begin{align*}
        \norm{\pi(x) + \pi(y)}_S
            &= \rho_S(\pi(x) + \pi(y), 0)
             = \rho_S(\pi(x), \pi(-y)) \\
            &\leq \rho_S(\pi(x), 0) + \rho_S(0, \pi(-y))
             = \norm{\pi(x)}_S + \norm{\pi(y)}_S,
    \end{align*}
    %
    so $\norm{\,\cdot\,}_S$ is a semi-norm. The rest of the claims follow by \cref{thm:coset-space-metric}.
\end{proof}


\nocite{*}

\printbibliography


\end{document}
