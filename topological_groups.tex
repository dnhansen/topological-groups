% Document setup
\documentclass[article, a4paper, 11pt, oneside]{memoir}
\usepackage[utf8]{inputenc}
\usepackage[T1]{fontenc}
\usepackage[UKenglish]{babel}

% Document info
\newcommand\doctitle{Topological Groups}
\newcommand\docauthor{Danny Nygård Hansen}

% Formatting and layout
\usepackage[autostyle]{csquotes}
\usepackage[final]{microtype}
\usepackage{xcolor}
\frenchspacing
\usepackage{latex-sty/articlepagestyle}
\usepackage{latex-sty/articlesectionstyle}

% Fonts
\usepackage[largesmallcaps]{kpfonts}
\DeclareSymbolFontAlphabet{\mathrm}{operators} % https://tex.stackexchange.com/questions/40874/kpfonts-siunitx-and-math-alphabets
\linespread{1.06}
\let\mathfrak\undefined
\usepackage{eufrak}
\usepackage{inconsolata}
\usepackage{amssymb}

% Hyperlinks
\usepackage{hyperref}
\definecolor{linkcolor}{HTML}{4f4fa3}
\hypersetup{%
	pdftitle=\doctitle,
	pdfauthor=\docauthor,
	colorlinks,
	linkcolor=linkcolor,
	citecolor=linkcolor,
	urlcolor=linkcolor,
	bookmarksnumbered=true
}

% Equation numbering
\numberwithin{equation}{chapter}

% Footnotes
\footmarkstyle{\textsuperscript{#1}\hspace{0.25em}}

% Mathematics
\usepackage{latex-sty/basicmathcommands}
\usepackage{latex-sty/framedtheorems}
\usepackage{latex-sty/topologycommands}
\usepackage{tikz-cd}
\usetikzlibrary{babel}

% Lists
\usepackage{enumitem}
\setenumerate[0]{label=\normalfont(\arabic*)}

% Bibliography
\usepackage[backend=biber, style=authoryear, maxcitenames=2, useprefix]{biblatex}
\addbibresource{references.bib}

% Title
\title{\doctitle}
\author{\docauthor}

\newcommand{\calT}{\mathcal{T}}


% Section style -- add to section style .sty?
\setsubsecheadstyle{\normalfont\itshape}


% Preimage -- to be added to mathcommands .sty
\newcommand{\preim}{^{-1}}


\begin{document}

\maketitle

\chapter{Introduction}

The purpose of these notes is to clarify some of the basic properties of topological groups. We in particular hope to shed light on the reasonability of the common assumption that topological groups be Hausdorff.

Most of the results below can be found in various forms in a variety of books or articles (see the references), but I have not come across a resource that develops this theory in precisely this way. One example is the $T_0$-identification of a topological group, as we will see below.


\chapter{Definitions and basic properties}

\begin{definition}[Topological groups]
    \label{def:topological_group}
    A \emph{topological group} is a triple $(G, \mu, \calT)$ such that $(G, \mu)$ is a group, $(G, \calT)$ is a topological space, and both the multiplication $\mu \colon G \prod G \to G$ and the inversion map $\iota \colon G \to G$ given by $g \mapsto g\inv$ are continuous.
\end{definition}
%
In the sequel we will simply write $G$, and it will be clear from context whether $G$ is to be thought of as a set, group, topological space or topological group. It is custom to assume that a topological group be $T_1$ (or Hausdorff, which is actually implied by it being $T_1$, as will follow from \cref{thm:topological_group_regular}), but we omit this assumption since our focus is precisely on the basic topological properties of topological groups.

\newcommand{\leftmult}{\lambda}
\newcommand{\rightmult}{\rho}

Let $G$ be a topological group and let $g \in G$. We define maps $\leftmult_g, \rightmult_g \colon G \to G$ by $\leftmult_g(h) = gh$ and $\rightmult_g(h) = hg$; that is, $\leftmult_g$ and $\rightmult_g$ are given by left- and right-multiplication by $g$, respectively. Both of these are homeomorphisms, with $\leftmult_g\inv = \leftmult_{g\inv}$ and $\rightmult_g\inv = \rightmult_{g\inv}$. Hence if $a, b \in G$, then there is a homeomorphism on $G$ that takes $a$ to $b$, namely $\leftmult_{ba\inv}$. Thus a topological group is \emph{homogeneous} as a topological space.

We begin by collecting some basic properties of topological groups:

\begin{proposition}[Properties of topological groups]
    Let $G$ be a topological group.
    %
    \begin{enumprop}
        \item \label{enum:closure_of_product} If $A,B \subseteq G$, then $\closure{A} \closure{B} \subseteq \closure{AB}$.
        \item \label{enum:closure_of_subgroup} If $H$ is a subgroup of $G$, then so is $\closure{H}$.
        \item \label{enum:open_subgroup_is_closed} Every open subgroup of $G$ is closed.
        \item \label{enum:product_open_set} If $A,U \subseteq G$ with $U$ open, then $AU$ and $UA$ are open.
    \end{enumprop}
\end{proposition}

\begin{proof}
    We give two proofs of \subcref{enum:closure_of_product}. Since the multiplication $\mu \colon G \prod G \to G$ is continuous we get
    %
    \begin{equation*}
        \mu(\closure{A}, \closure{B})
            = \mu(\closure{A} \prod \closure{B})
            = \mu(\closure{A \prod B})
            \subseteq \closure{\mu(A \times B)}
            = \closure{\mu(A, B)}
            = \closure{AB}.
    \end{equation*}
    %
    Alternatively, consider $a \in \closure{A}$ and $b \in \closure{B}$. If $U$ is a neighbourhood of $ab$, then by continuity of multiplication there are neighbourhoods $V_1$ and $V_2$ of $a$ and $b$ respectively such that $V_1 V_2 \subseteq U$. Picking $x \in A \intersect V_1$ and $y \in B \intersect V_2$ we find that $xy \in AB \intersect U$, so $ab \in \closure{AB}$.

    Let $H$ be a subgroup of $G$. By \subcref{enum:closure_of_product} we get
    %
    \begin{equation*}
        \closure{H} \closure{H} \subseteq \closure{HH} = \closure{H},
    \end{equation*}
    %
    so $\closure{H}$ is closed under multiplication. It is also closed under taking inverses, since the inverse map $\iota$ is a homeomorphism. This proves \subcref{enum:closure_of_subgroup}. Alternatively, this claim is an easy consequence of basic properties of nets.

    Now let $H$ be an open subgroup. Since $G$ is the disjoint union of the left cosets of $H$, we have
    %
    \begin{equation*}
        G \setminus H
            = \bigunion_{g \in G \setminus H} gH.
    \end{equation*}
    %
    Since the cosets $gH$ are open, $G \setminus H$ is also open. This proves \subcref{enum:open_subgroup_is_closed}.

    To prove \subcref{enum:product_open_set}, notice that e.g.
    %
    \begin{equation*}
        AU = \bigunion_{g \in A} gU,
    \end{equation*}
    %
    so $AU$ is a union of open sets.
\end{proof}

We next remark that the assumption that $G$ be $T_1$ can be weakened:

\begin{proposition}[$T_0$ implies $T_1$]
    \label{thm:T0_implies_T1}
    Let $G$ be a topological group. If $G$ is $T_0$, then it is in fact $T_1$.
\end{proposition}

\newcommand{\calN}{\mathcal{N}}
\DeclarePairedDelimiter{\nhoodfilteraux}{(}{)}
% \newcommand{\nhoodfilter}[1]{\calN\nhoodfilteraux{#1}}
\newcommand{\nhoodfilter}[1]{\calN_{#1}}

\begin{proof}
    Assume that $G$ is $T_0$. By homogeneity it suffices to show that the singleton $\{e\}$ containing the identity $e \in G$ is closed. We show that $G \setminus \{e\}$ is a neighbourhood\footnotemark{} of all its points. Let $g \neq e$. Since $G$ is $T_0$, either $G \setminus \{e\}$ is a neighbourhood of $g$ or $G \setminus \{g\}$ is a neighbourhood of $e$. In the first case we are done, so assume the latter. The homeomorphism $\leftmult_{g\inv}$ maps $G \setminus \{g\}$ to $G \setminus \{e\}$, so the latter set is a neighbourhood of $g\inv$. But the inversion map $\iota$ is a homeomorphism that maps $G \setminus \{e\}$ to itself, so this is also a neighbourhood of $g$. This proves the claim.
\end{proof}
\footnotetext{If $X$ is a topological space and $A \subseteq X$, then we say that a set $N \subseteq X$ is a \emph{neighbourhood} of $A$ if there is an open set $U$ in $X$ such that $A \subseteq U \subseteq N$. The family of neighbourhoods of a set $A$ is called the \emph{neighbourhood filter of $A$} and is denoted $\nhoodfilter{A}$. If $A = \{x\}$ is a singleton we also write $\nhoodfilter{x}$ and call $N$ a neighbourhood of $x$.}

Recall that a topological space $X$ is called \emph{regular} if it is possible to separate any point from any closed set by disjoint open sets. Our next order of business is to show that any topological group is regular. Before proving this we need some terminology and a lemma:

If $G$ is a topological group and $A \subseteq G$, then we write
%
\begin{equation*}
    A\inv = \iota(A) = \set{a\inv}{a \in A}.
\end{equation*}
%
A subset $A$ is called \emph{symmetric} if $A = A\inv$. Notice that since $\iota$ is a homeomorphism, $A$ is open (closed) if and only if $A\inv$ is open (closed).

\begin{lemma}
    \label{thm:symmetric_nhood_of_e}
    Let $G$ be a topological group. If $U$ is a neighbourhood of the identity $e$, then there is a symmetric neighbourhood $V$ of $e$ such that $VV \subseteq U$.
\end{lemma}

\begin{proof}
    Since multiplication is continuous and $ee = e$, there are neighbourhoods $V_1$ and $V_2$ of $e$ such that $V_1 V_2 \subseteq U$. If we let
    %
    \begin{equation*}
        V = V_1 \intersect V_2 \intersect V_1\inv \intersect V_2\inv,
    \end{equation*}
    %
    then $V$ has the desired properties.
\end{proof}


\begin{proposition}[Regularity of topological groups]
    \label{thm:topological_group_regular}
    If $G$ is a topological group, $g \in G$, $A \subseteq G$ is closed and $g \not\in A$, then there exists a symmetric neighbourhood $V$ of $e$ such that $Vg$ and $VA$ are disjoint. In particular every topological group is regular, and every $T_0$ topological group is $T_3$.
\end{proposition}

\begin{proof}
    Since $g \not\in A$ we also have $e \not\in Ag\inv$. But $Ag\inv$ is closed, so by \cref{thm:symmetric_nhood_of_e} there is a symmetric neighbourhood $V$ of $e$ such that $VV \intersect Ag\inv = \emptyset$. It follows that $Vg \intersect VA = \emptyset$ as desired. This shows that $G$ is regular, and the final claim follows by \cref{thm:T0_implies_T1}.
\end{proof}


\chapter{Coset spaces and quotient groups}

\section{General properties of coset spaces}

If $H$ is a subgroup of a topological group $G$, we denote by $G/H$ the set of left cosets of $H$. Let $q \colon G \to G/H$ be the quotient map and equip $G/H$ with the quotient topology. We call $G/H$ a \emph{coset space} of $G$.

Notice that $q$ is in fact open: If $U \subseteq G$ is open, then $q\preim(q(U)) = UH$ is also open, so $q(U)$ is open since $G/H$ has the quotient topology coinduced by $q$.


\begin{proposition}[Properties of coset spaces]
    Let $G$ be a topological group and $H$ a subgroup.
    %
    \begin{enumprop}
        \item \label{enum:coset_space_regular} The coset space $G/H$ is regular.
        
        \item \label{enum:coset_space_T1} $G/H$ is $T_1$ (and hence $T_3$) if and only if $H$ is closed.
        
        \item \label{enum:coset_space_locally_compact} If $G$ is locally compact\footnotemark, then so is $G/H$.
    \end{enumprop}
\end{proposition}\footnotetext{We say that a topological space is locally compact if every point has a compact neighbourhood. There are many non-equivalent definitions of local compactness and this is the least restrictive one. If $H$ is closed then $G/H$ is Hausdorff, so all the usual definitions of local compactness are equivalent for $G/H$.}
%
Notice that \subcref{enum:coset_space_T1} does not assume any separation properties of $G$.

(The claim \subcref{enum:coset_space_regular} is the content of supplementary exercise 7(d) of \textcite[Chapter~2]{munkres}, except that in the exercise $H$ is assumed to be closed. I do not see where this assumption is used.)

\begin{proof}
    We first show that $G/H$ is regular. Let $q(x) \in G/H$, and let $B \subseteq G/H$ be a closed set. Then $A = q\preim(B)$ is closed, so by \cref{thm:topological_group_regular} there is a symmetric neighbourhood $V$ of $e$ in $G$ such that $Vx \intersect VA = \emptyset$. Since $A$ is a union of left cosets of $H$ we have $A = AH$, and a quick calculations shows that $VxH \intersect VA = \emptyset$. It follows by a similar calculation that $q(Vx)$ and $q(VA)$ are disjoint, and since $q$ is open these are neighbourhoods of $q(x)$ and $q(A) = B$ respectively.

    Next we show that $G/H$ is $T_1$ if and only if $H$ is closed. Note that fibres of $q$ are cosets of $H$, and a coset $gH = \leftmult_g(H)$ is closed if and only if $H$ is. But since $G/H$ carries the quotient topology, $gH$ is closed in $G$ if and only if $\{gH\}$ is closed in $G/H$.

    % Now assuming that $H$ is closed, we show that $G/H$ is Hausdorff. Let $xH$ and $yH$ be distinct (and hence disjoint) cosets. Then $xHy\inv$ is a closed set not containing $e$, so \cref{thm:symmetric_nhood_of_e} implies the existence of a symmetric neighbourhood $U$ of $e$ such that $UU \intersect xHy\inv = \emptyset$. It follows that
    % %
    % \begin{equation*}
    %     e
    %         \not\in U xHy\inv U
    %         = Ux H (Uy)\inv
    %         = (UxH)(UyH)\inv,
    % \end{equation*}
    % %
    % where we use that $U = U\inv$ and $H = HH$. That $e$ does not lie in the left-most set is easily proven e.g. by contraposition. It follows that $UxH$ and $UyH$ are disjoint, and since $q$ is open this implies that $q(Ux)$ and $q(Uy)$ are disjoint neighbourhoods of $xH$ and $yH$ in $G/H$. This proves \subcref{enum:coset_space_hausdorff}.

    To prove \subcref{enum:coset_space_locally_compact}, notice that if $K$ is a compact neighbourhood of $e$ in $G$, then $q(Kx)$ is a compact neighbourhood of $xH$ in $G/H$.
\end{proof}

% This proposition also furnishes a different proof that $T_1$ implies Hausdorff for topological groups: Simply let $H = \{e\}$, in which case $G \cong G/H$.


\section{Topological quotient groups}

If a subgroup $N$ of a topological group $G$ is normal, we expect that the usual group structure on the (algebraic) quotient group $G/N$ is compatible with the quotient topology. This is indeed the case:

\begin{theorem}[Topological quotient groups]
    If $N$ is a normal subgroup of a topological group $G$, then $G/N$ is a topological group.
\end{theorem}

\begin{proof}
    If $x,y \in G$ and $U$ is a neighbourhood of $(xN)(yN) = xyN$ in $G/N$, then continuity of multiplication in $G$ at $(x,y)$ implies the existence of neighbourhoods $V$ and $W$ of $x$ and $y$ respectively, such that $VW \subseteq q\preim(U)$. Since $q$ is surjective it follows that $q(V) q(W) \subseteq U$, and because $q$ is also open $q(V)$ and $q(W)$ are neighbourhoods of $xN$ and $yH$. Hence multiplication is continuous.

    Since the inversion map $\iota$ on $G/N$ is bijective, it suffices to show that it is open. Let $U \subseteq G/N$ be open and notice that, since $q$ is surjective,
    %
    \begin{equation*}
        \iota(U)
            = \iota \bigl( q(q\preim(U)) \bigr)
            = q \bigl( \iota(q\preim(U)) \bigr).
    \end{equation*}
    %
    Because $q\preim(U)$, and hence $\iota(q\preim(U))$, is open in $G$, it follows that $\iota(U)$ is open since $q$ is open.
\end{proof}

We now explore how a topological group $G$ that is \emph{not} $T_0$ can be made so by quotienting out by a particular subgroup. To do this justice we take a small detour:

\subsection{The $T_0$-identification}

Let $X$ be a topological space. We define an ordering on $X$ called the \emph{specialisation preorder} by letting $x \leq y$ if $x \in \closure{\{y\}}$ for $x,y \in X$. It is clear that $\leq$ is in fact a preorder, and so it determines an equivalence relation $\equiv$; that is, $x \equiv y$ if and only if $x \leq y$ and $y \leq x$.

It is easy to show that $x \leq y$ if and only if $\nhoodfilter{x} \subseteq \nhoodfilter{y}$.  If $x \equiv y$, then we say that $x$ and $y$ are \emph{topologically indistinguishable} since then $x$ and $y$ have the same neighbourhoods.

It is clear that $X$ is $T_0$ if and only if the relation $\equiv$ is trivial, that is if $x$ and $y$ are topologically indistinguishable precisely when $x = y$. The quotient space $X/{\equiv}$ is called the \emph{$T_0$-identification} or the \emph{Kolmogorov quotient} of $X$, and it is indeed $T_0$:

\begin{lemma}[$T_0$-identification]
    Let $X$ be a topological space, and let $q \colon X \to X/{\equiv}$ be the quotient map onto the $T_0$-identification of $X$. Then $q$ is an open map and $X/{\equiv}$ is $T_0$.
\end{lemma}

\begin{proof}
    We first show that the quotient map $q \colon X \to X/{\equiv}$ is open. Let $U$ be an open set of $X$, and let $x \in U$. If $x \equiv x'$, then $U$ is also a neighbourhood of $x'$, so $x' \in U$. It follows that $q(x) \subseteq U$, and so
    %
    \begin{equation*}
        U = \bigunion_{x \in U} q(x).
    \end{equation*}
    %
    Hence $U$ is a union of $\equiv$-equivalence classes, and thus it is saturated. It follows that $q(U)$ is open, so $q$ is an open map.

    Now assume that $x \not\equiv y$. Without loss of generality we may assume the existence of an element $U \in \nhoodfilter{x} \setminus \nhoodfilter{y}$. Since $q$ is open, $q(U)$ is a neighbourhood of $q(x)$ in $X/{\equiv}$. We claim that $q(y) \not\in q(U)$: If not, then since $U$ is a union of equivalence classes there is a $z \in U$ with $y \equiv z$. But then $y \in U$ which is a contradiction. Thus $X/{\equiv}$ is $T_0$.
\end{proof}


\begin{proposition}
    Let $G$ be a topological group. The subgroup $\closure{\{e\}}$ of $G$ is normal. Furthermore, the $\equiv$-equivalence classes are precisely the left cosets of $\closure{\{e\}}$. It follows that the quotient group $G / \closure{\{e\}}$ is just the $T_0$-identification $G/{\equiv}$ of $G$.
\end{proposition}

\begin{proof}
    The subgroup $\closure{\{e\}}$ is the smallest closed subgroup of $G$, hence it is normal since otherwise intersecting it with one of its conjugates yields a strictly smaller closed subgroup.

    It then suffices to show that, for $x,y \in G$, $x \equiv y$ if and only if $x \closure{\{e\}} = y \closure{\{e\}}$. But this is clear since e.g. $x \closure{\{e\}} = \closure{\{x\}}$ by continuity of multiplication on $G$.
\end{proof}

It is results like the above that lead to the common assumption that topological groups are $T_1$, since if it is not then we just quotient out by $\closure{\{e\}}$. One might justify this by arguing that if the topology and the group structure on $G$ are truly meant to be compatible, then elements of $G$ that are topologically indistinguishable should also be indistinguishable with respect to the group structure.

There is more to say about the $T_0$-identification that is not pertinent to this discussion. We refer to \textcite[Exercise~13C]{willard} and \textcite{pirttimäki2019}.


\nocite{*}

\printbibliography


\end{document}
